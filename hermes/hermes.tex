\documentclass{article}


%%%%%%%%%%%%%%%%%%%%%%%%%%%%%%%%%%%%%%%%%%%%%%%%%%%%%%%%%%%%%%%%%%%%%%%%%
\pagestyle{plain}                                                      %%
%%%%%%%%%% EXACT 1in MARGINS %%%%%%%                                   %%
\setlength{\textwidth}{6.5in}     %%                                   %%
\setlength{\oddsidemargin}{0in}   %% (It is recommended that you       %%
\setlength{\evensidemargin}{0in}  %%  not change these parameters,     %%
\setlength{\textheight}{8.5in}    %%  at the risk of having your       %%
\setlength{\topmargin}{0in}       %%  proposal dismissed on the basis  %%
\setlength{\headheight}{0in}      %%  of incorrect formatting!!!)      %%
\setlength{\headsep}{0in}         %%                                   %%
\setlength{\footskip}{.5in}       %%                                   %%
%%%%%%%%%%%%%%%%%%%%%%%%%%%%%%%%%%%%                                   %%
\newcommand{\required}[1]{\section*{\hfil #1\hfil}}                    %%
\renewcommand{\refname}{\hfil References Cited\hfil}                   %%
\bibliographystyle{plain}                                              %%
%%%%%%%%%%%%%%%%%%%%%%%%%%%%%%%%%%%%%%%%%%%%%%%%%%%%%%%%%%%%%%%%%%%%%%%%%

\usepackage{graphicx}

\pagestyle{plain}

\begin{document}

\large

\vbox{}
\begin{figure}[!ht]
%\hspace{-4mm}
\includegraphics[width=8cm]{logo.png}
\vspace{4mm}
\end{figure}

\centerline{\huge \bf Using Hermes in NCLab}
\vspace{6mm}
\noindent
Hermes is a widely used open-source C++ library for rapid development of adaptive $hp$-FEM and $hp$-DG solvers.
In NCLab it is available via its Python wrappers. The wrappers follow the original C++ 
functionality very closely, so after understanding basic naming conventions, you should
be able to create your own examples by looking at the original C++ tutorial examples
which are available at {\tt http://hpfem.org/hermes/}.

\section*{Source code of Python wrappers}

If you are not sure how some Hermes functionality is used in Python, 
you can look it up easily. The wrappers are open source and they are 
available at

\begin{verbatim}
https://github.com/hpfem/hermes-python.git.
\end{verbatim} 
The folder and file structure of this repository is identical to 
the structure of the C++ library which is located at 

\begin{verbatim}
https://github.com/hpfem/hermes.git.
\end{verbatim} 
In case that we missed to wrap something that you need, let us know
and we will fix it.

\section*{Naming conventions}

The following basic rules apply:

\begin{itemize}
\item C++ class {\tt X} is wrapped as {\tt PyX}. 
\item C++ template {\tt T<double>} is wrapped as {\tt PyTReal}.  
\item C++ template {\tt T<complex>} is wrapped as {\tt PyTComplex}.  
\item Names of methods are identical. 
\item Arguments of methods are identical. 
\end{itemize}

\section*{Displayed tutorial examples in NCLab}

Several examples from the original Hermes tutorial are
available as displayed projects. To clone them, launch
File Manager, and click on Project $\rightarrow$ Clone.
In the table that appears, look for projects whose names 
start with "Hermes - Tutorial - Example ..."

\section*{Using GE and ME with Hermes}

The original tutorial examples come with their own predefined meshes. 
However, it is easy to create custom 2D domains and meshes using the Geometry Editor (GE)
and Mesh Editor (ME) and pass them to Hermes -- or to your finite element 
program for that matter. To see how this is done, 
clone the displayed projects whose names start with "GE and ME". 

\section*{Hermes - Tutorial - Example A03}

Let us describe in more detail the Python version 
of the Hermes tutorial example A-linear/03-poisson. 

\subsection*{Weak forms}

We begin with defining weak forms. C++ version:

{\small
\begin{verbatim}
CustomWeakFormPoisson::CustomWeakFormPoisson(std::string mat_al, Hermes1DFunction<double>* lambda_al,
                                             std::string mat_cu, Hermes1DFunction<double>* lambda_cu,
                                             Hermes2DFunction<double>* src_term) : WeakForm<double>(1)
{
  // Jacobian forms.
  add_matrix_form(new DefaultJacobianDiffusion<double>(0, 0, lambda_al, mat_al));
  add_matrix_form(new DefaultJacobianDiffusion<double>(0, 0, lambda_cu, mat_cu));

  // Residual forms.
  add_vector_form(new DefaultResidualDiffusion<double>(0, lambda_al, mat_al));
  add_vector_form(new DefaultResidualDiffusion<double>(0, lambda_cu, mat_cu));
  add_vector_form(new DefaultVectorFormVol<double>(0, -src_term));
};

\end{verbatim}
}
\noindent
Python version:
{\small
\begin{verbatim}
# Define weak forms:
class PyCustomWeakFormPoisson(hermes2d.PyCustomWeakFormReal):
    def __init__(self, mat_al, lambda_al, mat_cu, lambda_cu, vol_src_term):
        # Define number of equations.
        self.super(1)

        # Jacobian forms.
        self.add_matrix_form(hermes2d.PyDefaultJacobianDiffusion(0, 0, lambda_al, mat_al))
        self.add_matrix_form(hermes2d.PyDefaultJacobianDiffusion(0, 0, lambda_cu, mat_cu))

        # Residual forms.
        self.add_vector_form(hermes2d.PyDefaultResidualDiffusion(0, lambda_al, mat_al))
        self.add_vector_form(hermes2d.PyDefaultResidualDiffusion(0, lambda_cu, mat_cu))
        self.add_vector_form(hermes2d.PyDefaultVectorFormVol(0, -src_term))
\end{verbatim}
}
\noindent
The line {\tt self.super(1)} calls the constructor of the superclass with parameter 
{\tt 1} which is the number of equations. Notice that {\tt HERMES\_ANY} is present 
in the C++ version but it is omitted in the wrapper. 

\subsection*{Loading mesh from string in XML format}

This is the default way in tutorial examples. First we initialize 
an instance of the Mesh class. C++ version:

{\small
\begin{verbatim}
Mesh mesh;
\end{verbatim}
}
\noindent
Python version:

{\small
\begin{verbatim}
mesh = hermes2d.PyMesh()
\end{verbatim}
}
\noindent
Then we initialize an instance of the XML mesh reader. C++ version:

{\small
\begin{verbatim}
MeshReaderH2DXML mloader;
\end{verbatim}
}
\noindent
Python version:

{\small
\begin{verbatim}
reader = hermes2d.PyMeshReaderH2DXML()
\end{verbatim}
}
\noindent
For the time being, mesh strings are stored directly in the 
worksheets, later we will move them to separate files as 
in the C++ version. The XML format is described in detail
in the C++ tutorial:

{\small
\begin{verbatim}
mesh_stream = '<?xml version="1.0" encoding="utf-8"?>\
<mesh:mesh xmlns:xsi="http://www.w3.org/2001/XMLSchema-instance"\
 xmlns:mesh="XMLMesh"\
 xmlns:element="XMLMesh"\
 xsi:schemaLocation="XMLMesh /usr/include/hermes2d/xml_schemas/mesh_h2d_xml.xsd">\
 <variables>\
   <variable name="a" value="1.0" />\
   <variable name="m_a" value="-1.0" />\
   <variable name="b" value="0.70710678118654757" />	\
 </variables>\
\
 <vertices>\
   <vertex x="0.0" y="m_a" i="0"/>\
   <vertex x="a" y="m_a" i="1"/>\
   <vertex x="m_a" y="0.0" i="2"/>\
   <vertex x="." y="0.0" i="3"/>\
   <vertex x="a" y="0.0" i="4"/>\
   <vertex x="m_a" y="a" i="5"/>\
   <vertex x="0.0" y="a" i="6"/>\
   <vertex x="b" y="b" i="7"/>\
 </vertices>\
\
 <elements>\
   <element:quad v1="0" v2="1" v3="4" v4="3" marker="Copper" />\
   <element:triangle v1="3" v2="4" v3="7" marker="Copper" />\
   <element:triangle v1="3" v2="7" v3="6" marker="Aluminum" />\
   <element:quad v1="2" v2="3" v3="6" v4="5" marker="Aluminum" />\
 </elements>\
\
 <edges>\
   <edge v1="0" v2="1" marker="Bottom" />\
   <edge v1="1" v2="4" marker="Outer" />\
   <edge v1="3" v2="0" marker="Inner" />\
   <edge v1="4" v2="7" marker="Outer" />\
   <edge v1="7" v2="6" marker="Outer" />\
   <edge v1="2" v2="3" marker="Inner" />\
   <edge v1="6" v2="5" marker="Outer" />\
   <edge v1="5" v2="2" marker="Left" />\
 </edges>\
\
 <curves>\
   <arc v1="4" v2="7" angle="45" />\
   <arc v1="7" v2="6" angle="45" />\
 </curves>\
</mesh:mesh>'
\end{verbatim}
}
\noindent
This string is read by the XML mesh reader and passed into 
the instance of the Mesh class. C++ version (reading from 
file {\tt domain.xml}):

{\small
\begin{verbatim}
MeshReaderH2DXML mloader;  
mloader.load("domain.xml", &mesh);
\end{verbatim}
}
\noindent
Python version (reading from string):
{\small
\begin{verbatim}
reader.load_stream(mesh_stream, mesh)
\end{verbatim}
}

\subsection*{Using NCLab's GE and ME with your own FEM code}

NCLab's Mesh Editor has XML output, so one way to connect your 
own FEM code to ME is to implement a method that reads 
meshes in this format. ME also provides mesh output in raw format
consisting of arrays of vertices, elements with material markers, 
boundary edges with boundary markers, and curves. See displayed 
projects whose names start with "GE and ME". 


\subsection*{Mesh refinement operations}

All manual mesh refinement operations available in Hermes (see the C++ tutorial)
can be also used in Python. For example, the following will perform a uniform 
mesh refinement. C++ version:

{\small
\begin{verbatim}
mesh.refine_all_elements();
\end{verbatim}
}
\noindent
Python version:

{\small
\begin{verbatim}
mesh.refine_all_elements()
\end{verbatim}
}

\subsection*{Creating constant Dirichlet condition on certain boundary markers}

C++ version:
{\small
\begin{verbatim}
DefaultEssentialBCConst<double> bc_essential(
    Hermes::vector<std::string>("Bottom", "Inner", "Outer", "Left"), FIXED_BDY_TEMP);
\end{verbatim}
}
\noindent
Python version:

{\small
\begin{verbatim}
bc = hermes2d.PyDefaultEssentialBCConstReal(["Bottom", "Inner", "Left", "Outer"], FIXED_BDY_TEMP)
\end{verbatim}
}

\subsection*{Adding all BCs into a container to be passed into Space}

Since there can be many different boundary conditions, all BCs are 
first put into a container which is then passed into Space. C++ version:
{\small
\begin{verbatim}
EssentialBCs<double> bcs(&bc_essential);
\end{verbatim}
}
\noindent
Python version:

{\small
\begin{verbatim}
bcs = hermes2d.PyEssentialBCsReal(bc)
\end{verbatim}
}

\subsection*{Creating finite element space}

This model problem is a linear second-order equation whose solution 
is continuous, and therefore we use the $H^1$ space. All other spaces
available in C++ Hermes -- $H$(curl), $H$(div) and $L^2$ -- are available  
in Python as well. C++ version:

{\small
\begin{verbatim}
H1Space<double> space(&mesh, &bcs, P_INIT);
\end{verbatim}
}
\noindent
Python version:

{\small
\begin{verbatim}
space = hermes2d.PyH1SpaceReal(mesh, bcs, P_INIT)
\end{verbatim}
}

\subsection*{Initializing weak formulation}

C++ version:

{\small
\begin{verbatim}
CustomWeakFormPoisson wf("Aluminum", new Hermes1DFunction<double>(LAMBDA_AL), 
                         "Copper", new Hermes1DFunction<double>(LAMBDA_CU), 
                         new Hermes2DFunction<double>(VOLUME_HEAT_SRC));
\end{verbatim}
}
\noindent
Python version:

{\small
\begin{verbatim}
wf = PyCustomWeakFormPoisson("Aluminum", LAMBDA_AL, \
         "Copper", LAMBDA_CU, VOLUME_HEAT_SRC)
\end{verbatim}
}

\subsection*{Initializing discrete problem}

C++ version:

{\small
\begin{verbatim}
DiscreteProblem<double> dp(&wf, &space);
\end{verbatim}
}
\noindent
Python version:

{\small
\begin{verbatim}
dp = hermes2d.PyDiscreteProblemReal(wf, space)
\end{verbatim}
}

\subsection*{Initializing a Solution}

C++ version:

{\small
\begin{verbatim}
Solution<double> sln;
\end{verbatim}
}
\noindent
Python version:

{\small
\begin{verbatim}
solution = hermes2d.PySolutionReal()
\end{verbatim}
}

\subsection*{Initializing Newton solver}

C++ version:

{\small
\begin{verbatim}
NewtonSolver<double> newton(&dp, matrix_solver);
\end{verbatim}
}
\noindent
Python version:

{\small
\begin{verbatim}
newton = hermes2d.PyNewtonSolverReal(dp)
\end{verbatim}
}

\subsection*{Solving via the Newton's method}

C++ version:

{\small
\begin{verbatim}
newton.solve();
\end{verbatim}
}
\noindent
Python version:

{\small
\begin{verbatim}
newton.solve(coef)
\end{verbatim}
}
\noindent
The fact that the initial coefficient vector {\tt} is omitted on the 
C++ level means that it is a zero vector by default. 

\subsection*{Translating resulting coefficient vector into a Solution}

C++ version:

{\small
\begin{verbatim}
Solution<double>::vector_to_solution(newton.get_sln_vector(), &space, &sln);
\end{verbatim}
}
\noindent
Python version:

{\small
\begin{verbatim}
hermes2d.PySolutionReal().vector_to_solution(newton.get_sln_vector(), space, solution)
\end{verbatim}
}












{\small
\begin{verbatim}


\end{verbatim}
}




{\small
\begin{verbatim}


\end{verbatim}
}




{\small
\begin{verbatim}


\end{verbatim}
}




{\small
\begin{verbatim}


\end{verbatim}
}




{\small
\begin{verbatim}


\end{verbatim}
}




{\small
\begin{verbatim}


\end{verbatim}
}




{\small
\begin{verbatim}


\end{verbatim}
}




{\small
\begin{verbatim}


\end{verbatim}
}




{\small
\begin{verbatim}


\end{verbatim}
}










\end{document}

