\documentclass{article}


%%%%%%%%%%%%%%%%%%%%%%%%%%%%%%%%%%%%%%%%%%%%%%%%%%%%%%%%%%%%%%%%%%%%%%%%%
\pagestyle{plain}                                                      %%
%%%%%%%%%% EXACT 1in MARGINS %%%%%%%                                   %%
\setlength{\textwidth}{6.5in}     %%                                   %%
\setlength{\oddsidemargin}{0in}   %% (It is recommended that you       %%
\setlength{\evensidemargin}{0in}  %%  not change these parameters,     %%
\setlength{\textheight}{8.5in}    %%  at the risk of having your       %%
\setlength{\topmargin}{0in}       %%  proposal dismissed on the basis  %%
\setlength{\headheight}{0in}      %%  of incorrect formatting!!!)      %%
\setlength{\headsep}{0in}         %%                                   %%
\setlength{\footskip}{.5in}       %%                                   %%
%%%%%%%%%%%%%%%%%%%%%%%%%%%%%%%%%%%%                                   %%
\newcommand{\required}[1]{\section*{\hfil #1\hfil}}                    %%
\renewcommand{\refname}{\hfil References Cited\hfil}                   %%
\bibliographystyle{plain}                                              %%
%%%%%%%%%%%%%%%%%%%%%%%%%%%%%%%%%%%%%%%%%%%%%%%%%%%%%%%%%%%%%%%%%%%%%%%%%

\usepackage{graphicx}

\pagestyle{plain}

\begin{document}

\large

\vbox{}
\begin{figure}[!ht]
%\hspace{-4mm}
\includegraphics[width=8cm]{logo.png}
\vspace{4mm}
\end{figure}

\centerline{\huge \bf Using Hermes in NCLab}
\vspace{6mm}
\noindent
Hermes is a C++ library for rapid development of adaptive $hp$-FEM / $hp$-DG solvers.
In NCLab it is used via its Python wrappers. The wrappers follow the original C++ 
functionality very closely, so after understanding basic naming conventions, it
is easy to write Python examples by looking at the original C++ tutorial examples.

\section*{Naming conventions for Python wrappers}

The following basic rules apply:

\begin{itemize}
\item C++ class {\tt X} is wrapped as {\tt PyX}. 
\item C++ template {\tt T<real>} is wrapped as {\tt PyTReal}.  
\item C++ template {\tt T<complex>} is wrapped as {\tt PyTComplex}.  
\item Names of methods are identical. 
\item Arguments of methods are identical. 
\end{itemize}

\section*{Displayed tutorial examples in NCLab}

Several examples from the original Hermes tutorial are
available as displayed projects. To clone them, launch
File Manager, and click on Project $\rightarrow$ Clone.
In the table that appears, look for projects whose names 
start with "Hermes - Tutorial - Example ..."

\section*{Hermes - Tutorial - Example A03}

For illustration, let us describe in more detail the Python version 
of the Hermes tutorial example A-linear/03-poisson. 

\subsection*{Weak forms}

We begin with defining weak forms. The C++ version reads:

{\small
\begin{verbatim}
CustomWeakFormPoisson::CustomWeakFormPoisson(std::string mat_al, Hermes1DFunction<double>* lambda_al,
                                             std::string mat_cu, Hermes1DFunction<double>* lambda_cu,
                                             Hermes2DFunction<double>* src_term) : WeakForm<double>(1)
{
  // Jacobian forms.
  add_matrix_form(new DefaultJacobianDiffusion<double>(0, 0, mat_al, lambda_al));
  add_matrix_form(new DefaultJacobianDiffusion<double>(0, 0, mat_cu, lambda_cu));

  // Residual forms.
  add_vector_form(new DefaultResidualDiffusion<double>(0, mat_al, lambda_al));
  add_vector_form(new DefaultResidualDiffusion<double>(0, mat_cu, lambda_cu));
  add_vector_form(new DefaultVectorFormVol<double>(0, HERMES_ANY, src_term));
};

\end{verbatim}
}
The Python version is:
{\small
\begin{verbatim}
# Define weak forms:
class PyCustomWeakFormPoisson(hermes2d.PyCustomWeakFormReal):
    def __init__(self, mat_al, lambda_al, mat_cu, lambda_cu, vol_src_term):
        # Define number of equations.
        self.super(1)

        # Jacobian forms.
        self.add_matrix_form(hermes2d.PyDefaultJacobianDiffusion(0, 0, lambda_al, mat_al))
        self.add_matrix_form(hermes2d.PyDefaultJacobianDiffusion(0, 0, lambda_cu, mat_cu))
        self.add_vector_form(hermes2d.PyDefaultResidualDiffusion(0, lambda_al, mat_al))

        # Residual forms.
        self.add_vector_form(hermes2d.PyDefaultResidualDiffusion(0, lambda_cu, mat_cu))
        self.add_vector_form(hermes2d.PyDefaultVectorFormVol(0, -vol_src_term))
\end{verbatim}
}


\begin{verbatim}


\end{verbatim}

\begin{verbatim}


\end{verbatim}

\begin{verbatim}


\end{verbatim}

\begin{verbatim}


\end{verbatim}

\begin{verbatim}


\end{verbatim}



\end{document}

