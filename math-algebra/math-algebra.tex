\documentclass{article}


%%%%%%%%%%%%%%%%%%%%%%%%%%%%%%%%%%%%%%%%%%%%%%%%%%%%%%%%%%%%%%%%%%%%%%%%%
\pagestyle{plain}                                                      %%
%%%%%%%%%% EXACT 1in MARGINS %%%%%%%                                   %%
\setlength{\textwidth}{6.5in}     %%                                   %%
\setlength{\oddsidemargin}{0in}   %% (It is recommended that you       %%
\setlength{\evensidemargin}{0in}  %%  not change these parameters,     %%
\setlength{\textheight}{8.5in}    %%  at the risk of having your       %%
\setlength{\topmargin}{0in}       %%  proposal dismissed on the basis  %%
\setlength{\headheight}{0in}      %%  of incorrect formatting!!!)      %%
\setlength{\headsep}{0in}         %%                                   %%
\setlength{\footskip}{.5in}       %%                                   %%
%%%%%%%%%%%%%%%%%%%%%%%%%%%%%%%%%%%%                                   %%
\newcommand{\required}[1]{\section*{\hfil #1\hfil}}                    %%
\renewcommand{\refname}{\hfil References Cited\hfil}                   %%
\bibliographystyle{plain}                                              %%
%%%%%%%%%%%%%%%%%%%%%%%%%%%%%%%%%%%%%%%%%%%%%%%%%%%%%%%%%%%%%%%%%%%%%%%%%

\usepackage{graphicx}

\pagestyle{empty}

\begin{document}

\large

\vbox{}
\begin{figure}[!ht]
%\hspace{-4mm}
\includegraphics[width=8cm]{img/logo.png}
\vspace{29mm}
\end{figure}

\begin{figure}[!ht]
\begin{center}
%\hspace{-4mm}
\hspace{-20mm}
\includegraphics[width=14cm]{img/intro-frontpage.png}
\vspace{29mm}
\end{center}
\end{figure}

\centerline{\Huge \bf High-School Algebra with Sympy}

\vfill

\centerline{\Large www.nclab.com}

\newpage

%%%%%%%%%%%%%%%%%%%%%%%%%%%%%%%%%%%%%%%%%%%%%%%%%%%%%%%%%%%%%%%%%%%%%%%%%



\section*{}
\small
\input ../common/aboutnclab.tex

%\subsection*{Acknowledgement}
%This publication was created with the help of numerous freely 
%available web resources and tutorials related to Python, Scipy,
%Numpy, Pylab, Matplotlib, Sympy and other projects.

\normalsize

\newpage
%{\ }
\setcounter{tocdepth}{2}
\tableofcontents
%\pagestyle{plain}

\newpage

\pagestyle{plain}
\setcounter{page}{1}


%%%%%%%%%%%%%%%%%%%%%%%%%%%%%%%%%%%%%%%%%%%%%%%%%%%%%%%%%%%%%%%%%%%%%%%%%
\newpage

\pagestyle{plain}


\section{Introduction}

NCLab uses the Sympy library for algebra, calculus, differential equations, and related 
symbolic calculations. In the following we will show Sympy's basic usage. The library 
is much more powerful, and in order to learn more, we encourage you to visit Sympy's
home page {\tt http://sympy.org}. 

You can either type the 
following examples yourself (recommended) or clone the displayed project 
"Math - Tutorial - 07 - High School Algebra" through the File Manager's
{\em Project} menu. Remember that after launching a Python worksheet and 
typing text into an input cell, you need to click on the blue arrow button 
or on "run" under the input cell to have the script sent to the cloud and processed.

\section{Declaring symbolical variables}

We start by importing the Sympy library:

\begin{verbatim}
from sympy import *
\end{verbatim}
Then we need to declare the names of symbolic variables (otherwise Sympy would 
view those symbols as undefined constants). If we know that we are going to work 
with expressions and functions of $x$, $y$ and/or $z$, we may declare these symbols 
at once:
\begin{verbatim}
x, y, z = symbols("xyz")
\end{verbatim}
With this, we are ready to go!

\section{Entering an expression}
Algebraic expressions are represented using text strings that resemble 
computer code (well, they {\em are} computer code). For example, the expression 
$$
  (x + y)^2 (x + 1)
$$ 
needs to be entered as
\begin{verbatim}
(x + y)**2 * (x + 1)
\end{verbatim}
Longer expressions, when entered as one piece, may not be all that well readable. For example,
$$
\frac{(x + y)^2 (x + 1) + \frac{(x-5)(x-3)}{(x-4)^2} + (x + y)(x - 2)(x + y)^3}{(x-1)(x+2))}
$$ 
would be entered as 
\begin{verbatim}
((x + y)**2 * (x + 1) + (x - 5)*(x - 3)/(x - 4)**2 
+ (x + y)*(x - 2)*(x + y)**3)/((x - 1)*(x + 2))
\end{verbatim}
The same can be written more elegantly using simpler expressions:
\begin{verbatim}
A = (x + y)**2 * (x + 1)
B = (x - 5)*(x - 3)/(x - 4)**2
C = (x + y)*(x - 2)*(x + y)**3
D = (x - 1)*(x + 2)
(A + B + C)/D
\end{verbatim}
Do not forget that when a value is assigned to a variable (such as {\tt A = (x + y)**2 * (x + 1)}), nothing
is printed. To have a value of {\tt A} printed, you have multiple options including\\
\\
{\tt A}\\
{\tt print A}\\
{\tt print A = ", A}\\
\\
etc.

\section{Pretty printing}
An expression can be printed in a prettier way using the command {\tt pprint} (stands for {\em pretty print}):
\begin{verbatim}
pprint( (x+y)**2 * (x + 1) )
\end{verbatim}
Result:
\begin{verbatim}
       2
(x + y)  * (1 + x)
\end{verbatim}
Another example would be
\begin{verbatim}
pprint((x**2 + y**2 + z**2)/(x*y*z))
\end{verbatim}
Result:
\begin{verbatim}
 2   2   2
x + y + z
----------
  x*y*z   
\end{verbatim}

\section{Expanding expressions}

To expand an expression, we use the command {\tt expand}:
\begin{verbatim}
A = expand( (x+y)**2 * (x+1) )
pprint(A)
\end{verbatim}
Result:
\begin{verbatim}
         2    2      2        2    3
2*x*y + x  + y  + x*y  + 2*y*x  + x 
\end{verbatim}

\section{Simplifying expressions}

In order to simplify an expression, we use the command {\tt simplify}:
\begin{verbatim}
simplify( 1/x + (x*sin(x) - 1)/x )
\end{verbatim}
Result:
\begin{verbatim}
sin(x)
\end{verbatim}
Another example:
\begin{verbatim}
A = simplify( 1/x + 1/y + 1/z )
pprint(A)
\end{verbatim}
Result:
\begin{verbatim}
x*y + x*z + y*z
---------------
     x*y*z     
\end{verbatim}

\section{Factoring polynomials}

This is done using the command {\tt factor}:
\begin{verbatim}
factor(x**4 - 3*x**2 + 1)
\end{verbatim}
Result:
\begin{verbatim}
(1 + x - x**2)*(1 - x - x**2)
\end{verbatim}
In order to get a more thorough factorization, one can do:
\begin{verbatim}
factor(x**4 - 3*x**2 + 1, modulus=5)
\end{verbatim}
Result:
\begin{verbatim}
(2 + x)**2*(2 - x)**2
\end{verbatim}

\section{Partial fractions}

Partial fractions decomposition is done via the {\tt apart} command:
\begin{verbatim}
A = apart((1 + x) / (1 - x + x**2), x)
pprint(A)
\end{verbatim}
Result:
\begin{verbatim}
              ___                 ___  
          I*\/ 3              I*\/ 3   
    1/2 + -------       1/2 - -------  
             2                   2     
- ----------------- - -----------------
                ___                 ___
            I*\/ 3              I*\/ 3 
  1/2 - x - -------   1/2 - x + -------
               2                   2   
\end{verbatim}

\section{Calculating roots of polynomials}

The command {\tt roots} can be used to calculate all roots of a polynomial:
\begin{verbatim}
roots(-15 - 13*x + 3*x**2 + x**3, x)
\end{verbatim}
Result:
\begin{verbatim}
{-5: 1, -1: 1, 3: 1}
\end{verbatim}
It also works in the complex case:
\begin{verbatim}
roots(x**2 + 1, x)
\end{verbatim}
Result:
\begin{verbatim}
{-I: 1, I: 1}
\end{verbatim}


\section{Solving linear equations and systems}

In order to solve an equation or a system of equations, we use the command {\tt solve}.
For example, in order to solve the equation 
$$
3x - 4 - 2 = 0,
$$
we type:
\begin{verbatim}
solve(3*x - 4 - 2, [x])
\end{verbatim}
Result:
\begin{verbatim}
[2]
\end{verbatim}
\noindent
Here is an example for a system of two equations:
\begin{verbatim}
solve([x + 5*y - 2, -3*x + 6*y - 15], [x, y])
\end{verbatim}
Result:
\begin{verbatim}
{x: -3, y: 1}
\end{verbatim}
And one more example for a system of three equations:
\begin{verbatim}
solve([x + 5*y - z - 2, -3*x + 6*y + z - 15, x + y + z], [x, y, z])
\end{verbatim}
Result:
\begin{verbatim}
{x: -3, y: 1}{x: -40/17, y: 19/17, z: 21/17}
\end{verbatim}
\noindent

\section{Solving polynomial equations}

Nonlinear equations can be solved using the same command {\tt solve}.
For example, in order to solve the equation 
$$
3x^3 - 4 = 2,
$$
we type:
\begin{verbatim}
A = solve([3*x**3 - 4 - 2], [x])
pprint(A)
\end{verbatim}
Result:
\begin{verbatim}
              3 ___   ___   3 ___         3 ___   ___   3 ___   
  3 ___     I*\/ 2 *\/ 3    \/ 2        I*\/ 2 *\/ 3    \/ 2    
[(\/ 2 ,), (------------- - -----,), (- ------------- - -----,)]
                  2           2               2           2     
\end{verbatim}

\section{Solving transcendental equations}

For example, let us solve the equation
$$
e^x + 1 = 0.
$$
This is done as follows:
\begin{verbatim}
solve(exp(x) + 1, x)
\end{verbatim}
Result:
\begin{verbatim}
[pi*I]
\end{verbatim}

%%%%%%%%%%%%%%%%%%%%%%%%%%%%%%%%%%%%%%%%%%%%%%%%%%%%%%%%%%%%%%%%%%%%%%%%%




\end{document}
