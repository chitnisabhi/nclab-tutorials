\documentclass{article}


%%%%%%%%%%%%%%%%%%%%%%%%%%%%%%%%%%%%%%%%%%%%%%%%%%%%%%%%%%%%%%%%%%%%%%%%%
\pagestyle{plain}                                                      %%
%%%%%%%%%% EXACT 1in MARGINS %%%%%%%                                   %%
\setlength{\textwidth}{6.5in}     %%                                   %%
\setlength{\oddsidemargin}{0in}   %% (It is recommended that you       %%
\setlength{\evensidemargin}{0in}  %%  not change these parameters,     %%
\setlength{\textheight}{8.5in}    %%  at the risk of having your       %%
\setlength{\topmargin}{0in}       %%  proposal dismissed on the basis  %%
\setlength{\headheight}{0in}      %%  of incorrect formatting!!!)      %%
\setlength{\headsep}{0in}         %%                                   %%
\setlength{\footskip}{.5in}       %%                                   %%
%%%%%%%%%%%%%%%%%%%%%%%%%%%%%%%%%%%%                                   %%
\newcommand{\required}[1]{\section*{\hfil #1\hfil}}                    %%
\renewcommand{\refname}{\hfil References Cited\hfil}                   %%
\bibliographystyle{plain}                                              %%
%%%%%%%%%%%%%%%%%%%%%%%%%%%%%%%%%%%%%%%%%%%%%%%%%%%%%%%%%%%%%%%%%%%%%%%%%

\usepackage{graphicx}

\pagestyle{empty}

\begin{document}

\large

\vbox{}
\begin{figure}[!ht]
%\hspace{-4mm}
\includegraphics[width=8cm]{img/logo.png}
\vspace{29mm}
\end{figure}

\begin{figure}[!ht]
\begin{center}
%\hspace{-4mm}
\hspace{-20mm}
\includegraphics[width=14cm]{img/intro-frontpage.png}
\vspace{29mm}
\end{center}
\end{figure}

\centerline{\Huge \bf Calculus with Sympy}

\vfill

\centerline{\Large www.nclab.com}

\newpage

%%%%%%%%%%%%%%%%%%%%%%%%%%%%%%%%%%%%%%%%%%%%%%%%%%%%%%%%%%%%%%%%%%%%%%%%%



\section*{}
\small
\input ../common/aboutnclab.tex

%\subsection*{Acknowledgement}
%This publication was created with the help of numerous freely 
%available web resources and tutorials related to Python, Scipy,
%Numpy, Pylab, Matplotlib, Sympy and other projects.

\normalsize

\newpage
%{\ }
\setcounter{tocdepth}{2}
\tableofcontents
%\pagestyle{plain}

\newpage

\pagestyle{plain}
\setcounter{page}{1}


%%%%%%%%%%%%%%%%%%%%%%%%%%%%%%%%%%%%%%%%%%%%%%%%%%%%%%%%%%%%%%%%%%%%%%%%%
\newpage

\pagestyle{plain}


\section{Introduction}

NCLab uses the Sympy library for algebra, calculus, differential equations, and related 
symbolic calculations. In the following we will show Sympy's basic usage. The library 
is much more powerful, and in order to learn more, we encourage you to visit Sympy's
home page {\tt http://sympy.org}. 

You can either type the 
following examples yourself (recommended) or clone the displayed project 
"Math - Tutorial - 08 - Calculus" through the File Manager's
{\em Project} menu. Remember that after launching a Python worksheet and 
typing text into an input cell, you need to click on the blue arrow button 
or on "run" under the input cell to have the script sent to the cloud and 
processed.

\section{Declaring symbolical variables}

We start by importing the Sympy library:

\begin{verbatim}
from sympy import *
\end{verbatim}
Then we need to declare the names of symbolic variables (otherwise Sympy would 
view those symbols as undefined constants). If we know that we are going to work 
with expressions and functions of $x$, $y$ and/or $z$, we may declare these symbols 
at once:
\begin{verbatim}
x, y, z = symbols("x,y,z")
\end{verbatim}
With this, we are ready to go!

\section{Calculating limits}

\noindent
For example, let us show how to calculate the limit
$$
\lim_{x \rightarrow 0} \frac{\sin(x) - x}{x^3}.
$$
This is done using the command {\tt limit}:
\begin{verbatim}
limit((sin(x) - x)/x**3, x, 0)
\end{verbatim}
Result:
\begin{verbatim}
-1/6
\end{verbatim}
We can also calculate limits as $x \rightarrow \infty$. Infinity is represented 
in Sympy as {\tt oo}:
\begin{verbatim}
limit((x + sin(1/x))/(1 + x^2), x, oo)
\end{verbatim}
Result:
\begin{verbatim}
0
\end{verbatim}

\section{Calculating derivatives from definition}

One can use the {\tt limit} command to calculate derivatives:
\begin{verbatim}
limit((tan(x+y)-tan(x))/y, y, 0)
\end{verbatim}
Result:
\begin{verbatim}
1 + tan(x)**2
\end{verbatim}

\section{Differentiation}

Taking derivatives is very simple using the command {\tt diff}. For example:
\begin{verbatim}
diff(sin(x), x)
\end{verbatim}
Result:
\begin{verbatim}
cos(x)
\end{verbatim}
Another example:
\begin{verbatim}
result = diff(log(x) / x**2, x)
pprint(result)
\end{verbatim}
Result:
\begin{verbatim}
  2*log(x)   1 
- -------- + --
      3       3
     x       x 
\end{verbatim}

\section{Higher derivatives}

Taking higher derivatives is also simple:
\begin{verbatim}
diff(sin(2*x), x, 2)
\end{verbatim}
Result:
\begin{verbatim}
-4*sin(2*x)
\end{verbatim}

\section{Finding local extrems of functions}

One can differentiate a function and plot it. Let's say, for example, that 
we want to plot the derivative of the function $f(x) = 2\sin(2\pi(x - 1/4))$
in the interval $(0, 2)$. This is done as follows:
\begin{verbatim}
from numpy import sin, pi
from pylab import *
x = arange(0, 2, 0.01)
y = diff(2*sin(2 * pi * (x - 1/4.)), x)
clf()
plot(x, y)
lab.show()
\end{verbatim}
Analogously, we can differentiate the function twice and look for inflection points:
\begin{verbatim}
from numpy import sin, pi
from pylab import *
x = arange(0, 2, 0.01)
y = diff(2*sin(2 * pi * (x - 1/4.)), x, 2)
clf()
plot(x, y)
lab.show()
\end{verbatim}
Indeed it is possible to calculate the extrema using the function {\tt solve}.

\section{Taylor series expansion}

For this we use the {\tt series} command:
\begin{verbatim}
A = 1/cos(x)
result = A.series(x, 0, 7)
pprint(result)
\end{verbatim}
Result:
\begin{verbatim}
     2      4       6          
    x    5*x    61*x           
1 + -- + ---- + ----- + O(x**7)
    2     24     720           
\end{verbatim}

\section{Integration}

For example, let us calculate 
$$
\int \frac{1}{\sqrt{1 + 2x^2}}\, \mbox{d} x.
$$
This is done as follows:
\begin{verbatim}
result = integrate(1/sqrt(1 + 2*x**2), x)
pprint(result)
\end{verbatim}
Result:
\begin{verbatim}
  ___      /    ___\
\/ 2 *asinh\x*\/ 2 /
--------------------
         2      
\end{verbatim}

\section{Computing definite integrals}

For example, let us calculate
$$
\int_0^{\infty} e^{-x}\, \mbox{d} x.
$$
This is done as follows:
\begin{verbatim}
integrate(exp(-x), (x, 0, oo))
\end{verbatim}
Result:
\begin{verbatim}
pi**(1/2)
\end{verbatim}
Another example:
\begin{verbatim}
integrate(cos(x), (x, -pi/2, pi/2))
\end{verbatim}
Result:
\begin{verbatim}
2
\end{verbatim}

\section{Computing improper integrals}

For example, let us calculate
$$
\int_{-\infty}^{\infty} e^{-x^2}\, \mbox{d} x.
$$
This is done as follows:
\begin{verbatim}
integrate(exp(-x**2), (x, -oo, oo))
\end{verbatim}
Result:
\begin{verbatim}
pi**(1/2)
\end{verbatim}




\end{document}
