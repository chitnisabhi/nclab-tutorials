\documentclass{article}


%%%%%%%%%%%%%%%%%%%%%%%%%%%%%%%%%%%%%%%%%%%%%%%%%%%%%%%%%%%%%%%%%%%%%%%%%
\pagestyle{plain}                                                      %%
%%%%%%%%%% EXACT 1in MARGINS %%%%%%%                                   %%
\setlength{\textwidth}{6.5in}     %%                                   %%
\setlength{\oddsidemargin}{0in}   %% (It is recommended that you       %%
\setlength{\evensidemargin}{0in}  %%  not change these parameters,     %%
\setlength{\textheight}{8.5in}    %%  at the risk of having your       %%
\setlength{\topmargin}{0in}       %%  proposal dismissed on the basis  %%
\setlength{\headheight}{0in}      %%  of incorrect formatting!!!)      %%
\setlength{\headsep}{0in}         %%                                   %%
\setlength{\footskip}{.5in}       %%                                   %%
%%%%%%%%%%%%%%%%%%%%%%%%%%%%%%%%%%%%                                   %%
\newcommand{\required}[1]{\section*{\hfil #1\hfil}}                    %%
\renewcommand{\refname}{\hfil References Cited\hfil}                   %%
\bibliographystyle{plain}                                              %%
%%%%%%%%%%%%%%%%%%%%%%%%%%%%%%%%%%%%%%%%%%%%%%%%%%%%%%%%%%%%%%%%%%%%%%%%%

\usepackage{graphicx}

\pagestyle{empty}

\begin{document}

\large

\vbox{}
\begin{figure}[!ht]
%\hspace{-4mm}
\includegraphics[width=8cm]{img/logo.png}
\vspace{29mm}
\end{figure}

\begin{figure}[!ht]
\begin{center}
%\hspace{-4mm}
\hspace{-20mm}
\includegraphics[width=14cm]{img/intro-frontpage.png}
\vspace{34mm}
\end{center}
\end{figure}

\centerline{\Huge \bf Differential Equations with Sympy}

\vfill

\centerline{\Large www.nclab.com}

\newpage

%%%%%%%%%%%%%%%%%%%%%%%%%%%%%%%%%%%%%%%%%%%%%%%%%%%%%%%%%%%%%%%%%%%%%%%%%



\section*{}
\small
\input ../common/aboutnclab.tex

%\subsection*{Acknowledgement}
%This publication was created with the help of numerous freely 
%available web resources and tutorials related to Python, Scipy,
%Numpy, Pylab, Matplotlib, Sympy and other projects.

\normalsize

\newpage
%{\ }
\setcounter{tocdepth}{2}
\tableofcontents
%\pagestyle{plain}

\newpage

\pagestyle{plain}
\setcounter{page}{1}


%%%%%%%%%%%%%%%%%%%%%%%%%%%%%%%%%%%%%%%%%%%%%%%%%%%%%%%%%%%%%%%%%%%%%%%%%
\newpage

\pagestyle{plain}


\section{Introduction}

NCLab uses the Sympy library for algebra, calculus, differential equations, and related 
symbolic calculations. In the following we will show Sympy's basic usage. The library 
is much more powerful, and in order to learn more, we encourage you to visit Sympy's
home page {\tt http://sympy.org}. 

You can either type the 
following examples yourself (recommended) or clone the displayed project 
"Math - Tutorial - 10 - Differential Equations" through the File Manager's
{\em Project} menu. Remember that after launching a Python worksheet and 
typing text into an input cell, you need to click on the blue arrow button 
or on "run" under the input cell to have the script sent to the cloud and processed.\\

\noindent
{\bf NOTE}: If you are looking for numerical methods to solve ODE, then
a vast amount is provided in the Scipy and Numpy libraries that can 
be imported in any Python worksheet by typing

\begin{verbatim}
from scipy import *
\end{verbatim}
and

\begin{verbatim}
from numpy import *
\end{verbatim}
respectively. We refer to the web pages of these
two projects for detailed tutorials.

\section{Declaring symbolical variables}

We start by importing the Sympy library:

\begin{verbatim}
from sympy import *
\end{verbatim}
Then we need to declare the names of symbolic variables (otherwise Sympy would 
view those symbols as undefined constants). If we know that we are going to work 
with expressions and functions of $x$, $y$ and/or $z$, we may declare these symbols 
at once:
\begin{verbatim}
x, y, z = symbols("x,y,z")
\end{verbatim}
With this, we are ready to go!

\section{Separable equations}

For example, let us solve the equation 
$$
  y'(x) + 9y(x) = 1.
$$
This is done as follows:
\begin{verbatim}
y = Function("f")
A = dsolve(diff(y(x), x) + 9*y(x) - 1, y(x))
pprint(A)
\end{verbatim}
Result:
\begin{verbatim}
/      9*x\      
|     e   |  -9*x
|C1 + ----|*e    
\      9  /    
\end{verbatim}
Another example: Let us solve the equation 
$$
  y'(x) = -xy(x).
$$
This is done as follows:
\begin{verbatim}
y = Function("f")
A = dsolve(diff(y(x), x) + y(x) * x, y(x))
pprint(A)
\end{verbatim}
Result:
\begin{verbatim}
      2
    -x 
    ---
     2 
C1*e   
\end{verbatim}

\section{Linear equations - homogeneous}
Linear equations are equations of the form 
$$
a(x) y'(x) + b(x) y(x) = g(x).
$$ 
They are called {\em homogeneous} if $g(x) = 0$.
An example:
$$
(x^2 - 9) y'(x) + x y(x) = 0.
$$
This is done as follows:
\begin{verbatim}
y = Function("f")
A = dsolve((x**2 - 9)*diff(y(x), x) + y(x) * x, y(x))
pprint(A)
\end{verbatim}
Result:
\begin{verbatim}
     C1    
-----------
   ________
  /      2 
\/  9 - x  
\end{verbatim}

\section{Linear equations - nonhomogeneous}

An example of a nonhomogeneous linear equations is
$$
(x^2 - 9) y'(x) + x y(x) = x^2.
$$
This is done as follows:
\begin{verbatim}
y = Function("f")
A = dsolve((x**2 - 9)*diff(y(x), x) + y(x) * x - x**2, y(x))
pprint(A)
\end{verbatim}
Result:
\begin{verbatim}
           /x\        ________
     9*asin|-|       /      2 
           \3/   x*\/  9 - x  
C1 - --------- + -------------
         2             2      
------------------------------
            ________          
           /      2           
         \/  9 - x            
\end{verbatim}

%\section{Laplace transform}



\section{Nonlinear equations}

For example, let us solve the equation 
$$
  y'(x) = x^2 y^2(x).
$$
This is done as follows:
\begin{verbatim}
y = Function("f")
A = dsolve(diff(y(x), x) + y(x)**2 * x**2, y(x))
pprint(A)
\end{verbatim}
Result:
\begin{verbatim}
1/(C1 + x**3/3)
\end{verbatim}

%%%%%%%%%%%%%%%%%%%%%%%%%%%%%%%%%%%%%%%%%%%%%%%%%%%%%%%%%%%%%%%%%%%%%%%%%





\end{document}
