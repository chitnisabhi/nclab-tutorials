\documentclass{article}


%%%%%%%%%%%%%%%%%%%%%%%%%%%%%%%%%%%%%%%%%%%%%%%%%%%%%%%%%%%%%%%%%%%%%%%%%
\pagestyle{plain}                                                      %%
%%%%%%%%%% EXACT 1in MARGINS %%%%%%%                                   %%
\setlength{\textwidth}{6.5in}     %%                                   %%
\setlength{\oddsidemargin}{0in}   %% (It is recommended that you       %%
\setlength{\evensidemargin}{0in}  %%  not change these parameters,     %%
\setlength{\textheight}{8.5in}    %%  at the risk of having your       %%
\setlength{\topmargin}{0in}       %%  proposal dismissed on the basis  %%
\setlength{\headheight}{0in}      %%  of incorrect formatting!!!)      %%
\setlength{\headsep}{0in}         %%                                   %%
\setlength{\footskip}{.5in}       %%                                   %%
%%%%%%%%%%%%%%%%%%%%%%%%%%%%%%%%%%%%                                   %%
\newcommand{\required}[1]{\section*{\hfil #1\hfil}}                    %%
\renewcommand{\refname}{\hfil References Cited\hfil}                   %%
\bibliographystyle{plain}                                              %%
%%%%%%%%%%%%%%%%%%%%%%%%%%%%%%%%%%%%%%%%%%%%%%%%%%%%%%%%%%%%%%%%%%%%%%%%%

\usepackage{graphicx}

\pagestyle{empty}

\begin{document}

\large

\vbox{}
\begin{figure}[!ht]
%\hspace{-4mm}
\includegraphics[width=8cm]{img/logo.png}
\vspace{29mm}
\end{figure}

\begin{figure}[!ht]
\begin{center}
\includegraphics[width=10cm]{img/frontpage.png}
\vspace{29mm}
\end{center}
\end{figure}

\centerline{\Huge \bf Matrix Algebra with Numpy and Scipy}

\vfill

\centerline{\Large www.nclab.com}

\newpage

%%%%%%%%%%%%%%%%%%%%%%%%%%%%%%%%%%%%%%%%%%%%%%%%%%%%%%%%%%%%%%%%%%%%%%%%%



\section*{}


%\subsection*{Acknowledgement}
%This publication was created with the help of numerous freely 
%available web resources and tutorials related to Python, Scipy,
%Numpy, Pylab, Matplotlib, Sympy and other projects.

\normalsize

\newpage
%{\ }
\setcounter{tocdepth}{2}
\tableofcontents
%\pagestyle{plain}

\newpage

\pagestyle{plain}
\setcounter{page}{1}


%%%%%%%%%%%%%%%%%%%%%%%%%%%%%%%%%%%%%%%%%%%%%%%%%%%%%%%%%%%%%%%%%%%%%%%%%
\newpage

\pagestyle{plain}


\section{Introduction}

All examples in this section are available in the displayed project 
"Math - Tutorial - 09 - Matrix Algebra". You can clone the project into 
your user account via the menu Project $\rightarrow$ Clone in the File 
Manager, and experiment with the examples on your own.
NCLab provides vector and matrix operations via the Numpy library,
and in particular through its module {\tt linalg}:

\begin{verbatim}
from numpy import *
from numpy.linalg import *
\end{verbatim}

\section{Vector and its norm}

Vectors are Numpy arrays. They can be arbitrarily long, and they
are defined as follows: 
\begin{verbatim}
u = array([3, 4, 0])
\end{verbatim}
The Euclidean norm of a vector is calculated using the function {\tt norm}:
\begin{verbatim}
norm(u)
\end{verbatim}
Result:
\begin{verbatim}
5.0
\end{verbatim}

\section{Multiplying vector with a number}

All operations with vectors and matrices are very intuitive. To multiply 
a vector with a constant, type:

\begin{verbatim}
c = 2.5
u = array([3, 4, 0])
print c*u
\end{verbatim}
Result:
\begin{verbatim}
[  7.5  10.    0. ]
\end{verbatim}

\section{Adding and subtracting vectors}

Vectors can be added and subtracted just as numbers:

\begin{verbatim}
u = array([1, 3, 2])
v = array([-1, 1, 1])
print "u + v =", u+v
print "u - v =", u-v
\end{verbatim}
Result:
\begin{verbatim}
u + v = [0 4 3]
u - v = [2 2 1]
\end{verbatim}

\section{Inner product of vectors}

The inner product of vectors (often called {\em dot product}) is calculated
using the function {\tt dot}:

\begin{verbatim}
u = array([1, 3, 2])
v = array([-1, 1, 1])
dot(u, v)
\end{verbatim}
Result:
\begin{verbatim}
4.0
\end{verbatim}

\section{Outer product of vectors}

The outer product of vectors is obtained via the function {\tt outer}:
\begin{verbatim}
u = array([1, 3, 2])
v = array([-1, 1, 1])
outer(u, v)
\end{verbatim}
Result:
\begin{verbatim}
array([[-1,  1,  1],
       [-3,  3,  3],
       [-2,  2,  2]])
\end{verbatim}

\section{Matrix}

Matrices are two-dimensional Numpy arrays:
\begin{verbatim}
A = array([[1, 2, 3], [0, -2, 1], [1, 1, 0]])
print "A ="
print A
\end{verbatim}
Result:
\begin{verbatim}
A =
[[ 1  2  3]
 [ 0 -2  1]
 [ 1  1  0]]
\end{verbatim}


\section{Identity matrix}

Identity matrix is defined using the command {\tt eye}:
\begin{verbatim}
eye(5)
\end{verbatim}
Result:
\begin{verbatim}
array([[ 1.,  0.,  0.,  0.,  0.],
       [ 0.,  1.,  0.,  0.,  0.],
       [ 0.,  0.,  1.,  0.,  0.],
       [ 0.,  0.,  0.,  1.,  0.],
       [ 0.,  0.,  0.,  0.,  1.]])
\end{verbatim}

\section{Diagonal matrix}

Diagonal matrices are obtained via the {\tt diag} command:
\begin{verbatim}
diag([1, 2, 3, 4])
\end{verbatim}
Result:
\begin{verbatim}
array([[1, 0, 0, 0],
       [0, 2, 0, 0],
       [0, 0, 3, 0],
       [0, 0, 0, 4]])
\end{verbatim}

\section{Matrix transpose}

In order to obtain the matrix transpose, do:
\begin{verbatim}
A.transpose()
\end{verbatim}
Result:
\begin{verbatim}
array([[ 1,  0,  1],
       [ 2, -2,  1],
       [ 3,  1,  0]])
\end{verbatim}

\section{Matrix times vector}

Multiplication of matrices with vectors is done using the {\tt dot} function:
\begin{verbatim}
dot(A, u)
\end{verbatim}
Result:
\begin{verbatim}
array([13, -4,  4])
\end{verbatim}

\section{Matrix times matrix}

The same function {\tt dot} is used also to multiply two matrices:
\begin{verbatim}
A = array([[1, 2, 3], [0, -2, 1], [1, 1, 0]])
B = array([[1, -2, 0], [0, -2, 1], [1, 1, 1]])
dot(A, B)
\end{verbatim}
Result:
\begin{verbatim}
array([[ 4, -3,  5],
       [ 1,  5, -1],
       [ 1, -4,  1]])
\end{verbatim}

\section{Matrix power}

By $n$th power of a matrix $A$ we mean the product
$$
\underbrace{A \ \mbox{times}\ A \ \mbox{times} \ \ldots \ \mbox{times} \ A}_{n \ {\small \mbox{times}}}
$$ 
The command for this is:
\begin{verbatim}
n = 7
matrix_power(A, n)
\end{verbatim}
Result:
\begin{verbatim}
array([[ 379,  380,  472],
       [ -28, -261,  260],
       [ 176,  204,  175]])
\end{verbatim}

\section{Determinant of a matrix}

Determinants are calculated usign the command {\tt det}:
\begin{verbatim}
det(A)
\end{verbatim}
Result:
\begin{verbatim}
7.0
\end{verbatim}

\section{Condition number of a matrix}

By condition number of a matrix we mean the product of its norm with the 
norm of its inverse. Simply speaking, if the condition number is low
(lowest possible is 1.0), then it is easy to solve the system of linear
algebraic equations numerically. If the condition number is large, then
one can expect significant roundoff errors and other problems. One uses
the command {\tt cond} for that:
\begin{verbatim}
cond(A)
\end{verbatim}
Result:
\begin{verbatim}
4.9899406541341529
\end{verbatim}

\section{Inverse of a matrix}

The inverse of a nonsingular square matrix is obtained using the command {\tt inv}:
\begin{verbatim}
inv(A)
\end{verbatim}
Result:
\begin{verbatim}
array([[-0.14285714,  0.42857143,  1.14285714],
       [ 0.14285714, -0.42857143, -0.14285714],
       [ 0.28571429,  0.14285714, -0.28571429]])
\end{verbatim}

\section{Eigenvalues and eigenvectors}

A complete set of eigenvalues and eigenvectors can be obtained using the command {\tt eig}.
Note: The first array in the result contains the eigenvalues, the remaining arrays are the
eigenvectors:
\begin{verbatim}
eig(A)
\end{verbatim}
Result:
\begin{verbatim}
(array([ 2.50701864, -1.22187616, -2.28514248]), 
 array([[-0.91251758, -0.83862568,  0.31769573],
       [-0.088601  ,  0.42989448, -0.9118476 ],
       [-0.39932634,  0.33451115,  0.26000649]]))
\end{verbatim}

\section{QR factorization}

QR factorization of a matrix is obtained using the {\tt qr} command:
\begin{verbatim}
qr(A)
\end{verbatim}
Result:
\begin{verbatim}
(array([[-0.70710678,  0.23570226, -0.66666667],
       [-0.        , -0.94280904, -0.33333333],
       [-0.70710678, -0.23570226,  0.66666667]]), 
 array([[-1.41421356, -2.12132034, -2.12132034],
       [ 0.        ,  2.12132034, -0.23570226],
       [ 0.        ,  0.        , -2.33333333]]))
\end{verbatim}

\section{Cholesky decomposition}

Cholesky decomposition of a symmetric positive definite matrix is 
calculated via the command {\tt cholesky}:
\begin{verbatim}
C = array([[2, -1, 0], [-1, 2, -1], [0, -1, 2]])
cholesky(C)
\end{verbatim}
Result:
\begin{verbatim}
array([[ 1.41421356, -0.        ,  0.        ],
       [-0.70710678,  1.22474487, -0.        ],
       [ 0.        , -0.81649658,  1.15470054]])
\end{verbatim}

\section{LU factorization}

To perform LU factorization of a matrix, we import the function {\tt lu\_factor}
from the linalg module of the Scipy library:
\begin{verbatim}
from scipy.linalg import lu_factor
# Note: the two matrices are stored in one 2D array.
lu_factor(A)
\end{verbatim}
Result:
\begin{verbatim}
(array([[ 1. ,  2. ,  3. ],
       [ 0. , -2. ,  1. ],
       [ 1. ,  0.5, -3.5]]), array([0, 1, 2], dtype=int32))
\end{verbatim}
Note: the resulting matrices $L$ and $U$ are stored in one two-dimensional array.

\section{Singular value decomposition}

Singular value decomposition (SVD) of a matrix is obtained via the command {\tt svd}:
\begin{verbatim}
svd(A)
\end{verbatim}
Result:
\begin{verbatim}
(array([[ 0.95842033, -0.22464965,  0.17596308],
       [-0.1460451 , -0.91591194, -0.37386649],
       [ 0.24515567,  0.3326227 , -0.9106376 ]]), 
 array([ 3.8626099 ,  2.34116338,  0.77407933]), 
 array([[ 0.31159657,  0.63534413,  0.70657301],
       [ 0.0461194 ,  0.73260469, -0.6790901 ],
       [-0.94909461,  0.24418887,  0.19897542]]))
\end{verbatim}


\section{Solving systems of linear algebraic equations}

System of linear algebraic equations with a matrix $A$ and right-hand 
side vector $u$ is solved via the {\tt solve} command:
\begin{verbatim}
solve(A, u)
\end{verbatim}
Result:
\begin{verbatim}
array([ 3.42857143, -1.42857143,  0.14285714])
\end{verbatim}



\section{Moore-Penrose pseudo inverse}

A singular matrix does not have an inverse, but one can calculate the Moore-Penrose
pseudo inverse instead:
\begin{verbatim}
D = array([[1, 2, 3], [0, -2, 1], [1, 0, 4]])
pinv(D)
\end{verbatim}
Result:
\begin{verbatim}
array([[ 0.04347826, -0.01449275,  0.02898551],
       [ 0.20289855, -0.28985507, -0.08695652],
       [ 0.07246377,  0.08695652,  0.15942029]])
\end{verbatim}

\section{Least square solution with a singular matrix}

It is impossible to solve a system of linear algebraic equations with a singular
matrix in the classical sense, but it is possible to do it in the least-squares
sense:
\begin{verbatim}
D = array([[1, 2, 3], [0, -2, 1], [1, 0, 4]])
lstsq(D, u)
\end{verbatim}
Result:
\begin{verbatim}
(array([ 0.05797101, -0.84057971,  0.65217391]), array([], 
 dtype=float64), 2, 
 array([  5.36811455e+00,   2.68017652e+00,   4.40614709e-17]))
\end{verbatim}


%%%%%%%%%%%%%%%%%%%%%%%%%%%%%%%%%%%%%%%%%%%%%%%%%%%%%%%%%%%%%%%%%%%%%%%%%


\end{document}
