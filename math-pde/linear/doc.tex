\documentclass{article}


%%%%%%%%%%%%%%%%%%%%%%%%%%%%%%%%%%%%%%%%%%%%%%%%%%%%%%%%%%%%%%%%%%%%%%%%%
\pagestyle{plain}                                                      %%
%%%%%%%%%% EXACT 1in MARGINS %%%%%%%                                   %%
\setlength{\textwidth}{6.5in}     %%                                   %%
\setlength{\oddsidemargin}{0in}   %% (It is recommended that you       %%
\setlength{\evensidemargin}{0in}  %%  not change these parameters,     %%
\setlength{\textheight}{8.5in}    %%  at the risk of having your       %%
\setlength{\topmargin}{0in}       %%  proposal dismissed on the basis  %%
\setlength{\headheight}{0in}      %%  of incorrect formatting!!!)      %%
\setlength{\headsep}{0in}         %%                                   %%
\setlength{\footskip}{.5in}       %%                                   %%
%%%%%%%%%%%%%%%%%%%%%%%%%%%%%%%%%%%%                                   %%
\newcommand{\required}[1]{\section*{\hfil #1\hfil}}                    %%
\renewcommand{\refname}{\hfil References Cited\hfil}                   %%
\bibliographystyle{plain}                                              %%
%%%%%%%%%%%%%%%%%%%%%%%%%%%%%%%%%%%%%%%%%%%%%%%%%%%%%%%%%%%%%%%%%%%%%%%%%

\usepackage{graphicx}

\pagestyle{empty}


\begin{document}

\large

\vbox{}
\vspace{-2cm}
\begin{figure}[!ht]
%\hspace{-4mm}
\includegraphics[width=8cm]{img/logo.png}
\vspace{4mm}
\end{figure}
\noindent
\begin{center}
{\Huge Solving Linear Second-Order PDE\\ with the Finite Element Method}\\[3mm]
Based on Hermes2D ({\tt http://hpfem.org/hermes})\\[6mm]
\end{center}
\section{Module Description}

This module is designed to solve general linear second-order
partial differential equations (PDE) of the form 

$$
-\sum_{i, j = 1}^d \frac{\partial}{\partial x_i} \left(a_{ij}
 \frac{\partial u}{\partial x_j}  \right) + \sum_{i=1}^d b_i \frac{\partial u}{\partial x_i}
+ cu = f 
$$  
where $d=2$ is the spatial dimension, and $a_{ij}$, $b_i$, $c$ and $f$ are constants,
which can be different in subdomains. One can prescribe Dirichlet, Neumann, and Newton 
(Robin) boundary conditions. The boundary conditions can be combined arbitrarily. 

\section{Interesting Special Cases}

With $a_{11} = a_{22} = 1$, $a_{12} = a_{21} = b_1 = b_2 = c = 0$ we obtain 
the {\em Poisson equation} 
$$
-\Delta u = f.
$$ 
This equation has several important applications 
in physics. It is used to model electrostatics ($u$ being the electric 
potential and $f$ the electric charge density divided by the electric permittivity), 
stationary heat transfer equation ($u$ being the temperature and $f$ the heat 
sources or losses), and other diffusive processes.

Setting moreover $f = 0$, one obtains the {\em Laplace equation}

$$
-\Delta u = 0
$$ 
which describes linear magnetostatics ($u$ being the scalar magnetic potential),
stationary wave equation ($u$ being the amplitude), and it can also be used 
to compute the shape of an elastic membrane (smallest surface) that spans 
a closed curve.

With $a_{11} = a_{22} = 1$, $a_{12} = a_{21} = b_1 = b_2 = f = 0$
and $c < 0$ one obtains the {\em Helmholtz equation}

$$
-\Delta u - k^2 u = 0
$$ 
where $k^2 = -c$ is the square of the wave number.

As the last example, with $a_{11} = a_{22} = D$, $a_{12} = a_{21} = 0$, $b_1 = v_1$, $b_2 = v_2$, and 
$f = c = 0$ one obtains the {\em stationary advection-diffusion equation}

$$
-\mbox{div}(D \nabla u) + {\bf v}\cdot \nabla u = 0.
$$ 
Here $u$ is the advected quantity, $D > 0$ is the diffusion coefficient, 
and ${\bf v} = (v_1, v_2)$ is the velocity of the flowing medium that is advecting 
the quantity $u$.

\section{Boundary Conditions}

One can split the boundary into subsets where different boundary conditions are 
prescribed:
\begin{itemize}
\item {\em Dirichlet conditions}: $u = u^*$ where $u^*$ is a constant.
\item {\em Neumann conditions}: 
$$
\sum_{i, j = 1}^d a_{ij} \frac{\partial u}{\partial x_j} \nu_i = c
$$
where $(\nu_1, \nu_2)$ is the unit outer normal vector to the domain's boundary, 
and $c$ is a constant.
\item {\em Newton (Robin) conditions}: 
$$
\sum_{i, j = 1}^d a_{ij} \frac{\partial u}{\partial x_j} \nu_i  + c_1 u = c_2
$$
where $(\nu_1, \nu_2)$ is the unit outer normal vector to the domain's boundary, 
and $c_1, c_2$ are constants.
\end{itemize}


\end{document}

