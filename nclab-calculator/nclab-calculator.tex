\documentclass{article}


%%%%%%%%%%%%%%%%%%%%%%%%%%%%%%%%%%%%%%%%%%%%%%%%%%%%%%%%%%%%%%%%%%%%%%%%%
\pagestyle{plain}                                                      %%
%%%%%%%%%% EXACT 1in MARGINS %%%%%%%                                   %%
\setlength{\textwidth}{6.5in}     %%                                   %%
\setlength{\oddsidemargin}{0in}   %% (It is recommended that you       %%
\setlength{\evensidemargin}{0in}  %%  not change these parameters,     %%
\setlength{\textheight}{8.5in}    %%  at the risk of having your       %%
\setlength{\topmargin}{0in}       %%  proposal dismissed on the basis  %%
\setlength{\headheight}{0in}      %%  of incorrect formatting!!!)      %%
\setlength{\headsep}{0in}         %%                                   %%
\setlength{\footskip}{.5in}       %%                                   %%
%%%%%%%%%%%%%%%%%%%%%%%%%%%%%%%%%%%%                                   %%
\newcommand{\required}[1]{\section*{\hfil #1\hfil}}                    %%
\renewcommand{\refname}{\hfil References Cited\hfil}                   %%
\bibliographystyle{plain}                                              %%
%%%%%%%%%%%%%%%%%%%%%%%%%%%%%%%%%%%%%%%%%%%%%%%%%%%%%%%%%%%%%%%%%%%%%%%%%

\usepackage{graphicx}

\pagestyle{empty}

\begin{document}

\large

\vbox{}
\begin{figure}[!ht]
%\hspace{-4mm}
\includegraphics[width=8cm]{img/logo.png}
\vspace{12mm}
\end{figure}

\begin{figure}[!ht]
\begin{center}
%\hspace{-4mm}
\includegraphics[width=14cm]{img/webgl.png}
\vspace{16mm}
\end{center}
\end{figure}

\centerline{\Huge \bf NCLab as a Graphing Calculator}

\vfill

\centerline{\Large www.nclab.com}

\newpage

%%%%%%%%%%%%%%%%%%%%%%%%%%%%%%%%%%%%%%%%%%%%%%%%%%%%%%%%%%%%%%%%%%%%%%%%%



\section*{}
\small
\input ../common/aboutnclab.tex

%\subsection*{Acknowledgement}
%This publication was created with the help of numerous freely 
%available web resources and tutorials related to Python, Scipy,
%Numpy, Pylab, Matplotlib, Sympy and other projects.

\normalsize

\newpage
%{\ }
\setcounter{tocdepth}{2}
\tableofcontents
%\pagestyle{plain}

\newpage

\pagestyle{plain}
\setcounter{page}{1}


%%%%%%%%%%%%%%%%%%%%%%%%%%%%%%%%%%%%%%%%%%%%%%%%%%%%%%%%%%%%%%%%%%%%%%%%%
\newpage

\pagestyle{plain}


\section{Introduction}

The advantage of simple calculators, such as the one shown in Fig. \ref{fig:xcalc}
is that they only have a few functions, and thus can be operated using a few buttons.

\begin{figure}[!ht]
\begin{center}
\includegraphics[width=0.4\textwidth]{img/xcalc.png}
\end{center}
%\vspace{-2mm}
\caption{Simple calculator.}
\label{fig:xcalc}
\end{figure}
\noindent
Compared to this calculator, NCLab provides an enormous functionality 
that would require hundreds of buttons. So instead of buttons,
in NCLab we use simple commands to get the results we want. The language used 
to talk to NCLab is called Python. Using Python is very intuitive, 
as we shall see in a moment.

\section{Launching a Python Project}

Python worksheet can be launched through the File Manager or through 
the Programming icon on Desktop. Let's do for example the latter. 
This will lanuch a new Python worksheet, as illustrated in Fig. 
\ref{fig:python}. 

\newpage

\begin{figure}[!ht]
\begin{center}
\includegraphics[width=\textwidth]{img/python.png}
\end{center}
%\vspace{-2mm}
\caption{Python worksheet.}
\label{fig:python}
\end{figure}
\noindent

\section{Cloning Displayed Projects}

All examples that we are going to work with in the following are also available 
as Displayed Projects. This means that you can clone them by going to the
File Manager's Project menu and clicking on {\em Clone}. This will launch 
a window with many displayed projects from various areas of programming,
math and computing. Look for projects titled such as "Math - Tutorial - 1 - Simple Arithmetic",
"Math - Tutorial - 2 - Commutative, Associative and Distributive Laws", etc.
After you locate a project that you would like to clone, click on it,
and then click on the button "Clone" at the bottom of the window. This will
create exact copy of that project in your account, and you can open it 
by clicking on it in the File Manager. You can change the project in any way 
you like, the changes will not affect the original Displayed Project. 

\section{Simple Arithmetic}

Let's say that we have not cloned the displayed project "Math - Tutorial - 1 - Simple Arithmetic"
and instead, we prefer to start from scratch.
So your project contains a single {\em input cell}. Let's write 
something simple into it, for example "1 + 2", and click 
the link "run" right under the input cell. This sends a request to 
the cloud, it is processed there, and an answer comes back instantly. 
It is displayed in a new (yellow) {\em output cell}, as shown in 
Fig. \ref{fig:1p3}.

\begin{figure}[!ht]
\begin{center}
\includegraphics[width=\textwidth]{img/1p2.png}
\end{center}
%\vspace{-2mm}
\caption{Evaluating the expression "1+2".}
\label{fig:1p3}
\end{figure}
\noindent
\noindent
In addition to input and output cells, you can use {\em text cells}
to keep your project commented. This is strongly recommended. In
order to add a new text cell above the input cell, click into
the input cell, and then in menu Edit choose "New text cell above
active cell". A new text cell appears, asking you to click there to
edit its contents. Do so, and write, for example, "**Adding numbers**".
The double stars are used to make a text between them bold face. 
Then click on the link "save" right under the text cell. The result 
is shown in Fig. \ref{fig:1p3r}.

\newpage


\begin{figure}[!ht]
\begin{center}
\includegraphics[width=\textwidth]{img/1p3r.png}
\end{center}
%\vspace{-2mm}
\caption{New descriptive text cell was added above the input cell.}
\label{fig:1p3r}
\end{figure}
\noindent
\noindent
If you like, you can click back into the input cell, change 
the numbers, and click the "run" link under the cell again. 
This will send a new request to the cloud and after the answer 
comes back, it is displayed in the existing output 
cell. 

Let's continue by creating a new empty input cell. To do this, click 
on the "add" below the last input cell. The new input cell will appear below the yellow 
output cell (it will never be placed between an input cell and its 
result). In there, we can experiment with other arithmetic operations 
including subtraction (try for example "5 - 3"), multiplication 
(try for example "3.21 * 7.45"). The results are shown in Fig. \ref{fig:1p3r2}.

\newpage

\begin{figure}[!ht]
\begin{center}
\includegraphics[width=\textwidth]{img/1p3r2.png}
\end{center}
%\vspace{-2mm}
\caption{Subtraction and multiplication.}
\label{fig:1p3r2}
\end{figure}
\noindent
At this point we may go to menu Edit and click on 
"Remove all output" -- this will free up some space.\\

\noindent
{\em BEWARE - Division of Two integers Always is an Integer}\\

Numbers such as "12" or "5" are integers. In Python, as well as 
in other major programming languages, the {\bf result of division of 
two integers always is an integer}. This means that anything beyond 
the decimal point in the result is erased! Let us see this in reality:

\newpage
\begin{figure}[!ht]
\begin{center}
\includegraphics[width=\textwidth]{img/div1.png}
\end{center}
%\vspace{-2mm}
\caption{Division of two integers always is an integer \& two ways to make division safe.}
\label{fig:div1}
\end{figure}
\noindent
Fig. \ref{fig:div1} shows that evaluating simply 12/5 leads to a wrong result. An easy fix is 
to convert one of the numbers into a float by appending a decimal point to it. Still
another way, also shown in  Fig. \ref{fig:div1}, is to convert one of the numbers into a float 
via the function {\tt float()}. The latter approach works even for the division of variables "a/b"
where one may not know exactly whether they are integers or floats. Whenever at least 
one of the numbers if float, the result is a float. \\

\noindent
If you are a mathematician -- yes, you are right. we forgot to discuss division by zero. 
It is useful to see how NCLab reports errors, so let's do it!

\newpage
\begin{figure}[!ht]
\begin{center}
\includegraphics[width=\textwidth]{img/divzero.png}
\end{center}
%\vspace{-2mm}
\caption{Error is reported when dividing by zero.}
\label{fig:divzero}
\end{figure}
\noindent
Since the error message in Fig. \ref{fig:divzero} mentioned modulo, we added there one more 
input cell demonstrating how modulo is done -- using the \% symbol.

\section{Commutative, Associative and Distributive Laws}

The validity of these laws can be verified easily on examples. One just enters
a code similar to 

\begin{verbatim}
a = 2
b = 3
print "a + b =", a + b
print "b + a =", b + a
\end{verbatim}
and clicks on the link "run" under the input cell. The output 
is shown in Fig. \ref{fig:commut}. 

\begin{figure}[!ht]
\begin{center}
\includegraphics[width=0.6\textwidth]{img/commut.png}
\end{center}
\vspace{-4mm}
\caption{Commutative Law holds!}
\label{fig:commut}
%\vspace{-6mm}
\end{figure}




\section{List of Elementary Functions}

In order to calculate square roots, exponentials, sins, cosins, tangents, and all other 
simple functions, the best way is to import Numpy as shown in Fig. \ref{fig:fns}. Numpy 
is a standard Python library for numerical computations.

\begin{figure}[!ht]
\begin{center}
\includegraphics[width=\textwidth]{img/fns.png}
\end{center}
\vspace{-4mm}
\caption{Importing Numpy and using elementary functions.}
\label{fig:fns}
%\vspace{-6mm}
\end{figure}

\newpage
\noindent
Elementary functions that one can import from Numpy include:\\

%{\small
\begin{center}
\begin{tabular}{|l|l|}
\hline
abs($x$) &  absolute value of $x$\\
arccos($x$) &  inverse cosine of $x$ \\
arccosh($x$) &  inverse hyperbolic cosine of $x$ \\
arcsin($x$) & inverse sine of $x$ \\
arcsinh($x$) & inverse hyperbolic sine of $x$ \\
arctan($x$) & inverse tangent of $x$ \\
arctanh($x$) & inverse hyperbolic tangent of $x$ \\
arctan2($x_1$, $x_2$) & arc tangent of $x_1/x_2$ choosing the quadrant correctly \\
cos($x$) & cosine of $x$ \\
cosh($x$) & hyperbolic tangent of $x$ \\
exp($x$) & $e^x$ \\
log($x$) & natural logarithm of $x$ \\
pow($a$, $b$) & $a^b$ (same as "a**b")\\
sin($x$) & sine of $x$ \\
sinh($x$) & hyperbolic sine of $x$ \\
sqrt($x$) & square root of $x$ \\
tan($x$) & tangent of $x$\\
tanh($x$) & hyperbolic tangent of $x$ \\
\hline
\end{tabular}
\end{center}
%}
\vspace{4mm}
\noindent
For a complete overview of functions provided by Numpy we recommend the 
web page \\ {\tt http://www.scipy.org/Numpy\_Functions\_by\_Category}.

%%%%%%%%%%%%%%%%%%%%%%%%%%%%%%%%%%%%%%%%%%%%%%%%%%%%%%%%%%%%%%%%%%%%%%%%

\section{Plotting Functions of One Variable}\label{plotting}

Plotting can be done via the Pylab library. Pylab is a standard Python library that is 
superior in many ways to MATLAB (expensive commercial product). To use Pylab, login to NCLab and launch a Python 
worksheet first.

The Pylab {\tt plot} command takes two
lists: $x$-coordinates and $y$-coordinates of points on a curve. Between the 
points, the curve is interpolated linearly. Let us illustrate this on a simple 
example with just five points [0, 0], [1, 2], [2, 0.5], [3, 2.5] and [4, 0]:

\begin{verbatim}
from pylab import *
x = [0,0, 1.0, 2.0, 3.0, 4.0]
y = [0.0, 2.0, 0.5, 2.5, 0]
clf()
plot(x, y)
lab.show()
\end{verbatim}
The commands {\tt clf()}, {\tt plot()} and {\tt lab.show()} do clear the canvas, 
plot the graph, and show the graph, respectively.
The output is shown in Fig. \ref{fig:plot}.


\begin{figure}[!ht]
\begin{center}
\hbox{}
\hspace{-6mm}
\includegraphics[width=0.56\textwidth]{img/plot.png}
\end{center}
\vspace{-2mm}
\caption{Piecewise-linear curve with five points.}
\label{fig:plot}
%\vspace{-1cm}
\end{figure}
\noindent
In the following we will discuss more options and show some useful techniques.
Let's say, for example, that we want to plot the function $f(x) = \sin(x)$
in the interval $(0, 2\pi)$. The array of $x$-coordinates of equidistant points 
between 0 and $\pi$ with step 0.05 can be created easily using the command {\tt arange}:

\begin{verbatim}
from numpy import *
x = arange(0, 2*pi, 0.05)
\end{verbatim}
Changing the step size will change the resolution - with a smaller step the resolution will 
be finer and vice-versa. Next, the array of $y$-coordinates of the points is obtained via

\begin{verbatim}
y = sin(x)
\end{verbatim}
The last part we already know:

\begin{verbatim}
clf()
plot(x, y)
lab.show()
\end{verbatim}
\noindent
The output is shown in Fig. \ref{fig:plot1}.\\[-7mm]

\begin{figure}[!ht]
\begin{center}
\includegraphics[width=0.6\textwidth]{img/plot1.png}
\end{center}
\vspace{-6mm}
\caption{Plotting $\sin(x)$ in interval $(0, 2\pi)$ with subdivision step 0.05.}
\label{fig:plot1}
\vspace{-2mm}
\end{figure}
\noindent
The plot can be made nicer by adding a label, and also the color 
and the line style can be changed. Let us start with adding a label:

\begin{verbatim}
lb = "Solid blue line"
plot(x, y, 'b-', label = lb)
legend()
lab.show()
\end{verbatim}
The output is shown in Fig. \ref{fig:plot2}.\\[-7mm]

\begin{figure}[!ht]
\begin{center}
\includegraphics[width=0.6\textwidth]{img/plot2.png}
\end{center}
\vspace{-6mm}
\caption{Adding a label.}
\label{fig:plot2}
\vspace{-1cm}
\end{figure}
\newpage
\noindent
Next let us change the color to red and line style to dashed: 

\begin{verbatim}
lb = "Dashed red line"
clf()
plot(x, y, 'r--', label = lb)
legend()
lab.show()
\end{verbatim}
The output is shown in Fig. \ref{fig:plot3}.



\begin{figure}[!ht]
\begin{center}
\includegraphics[width=0.6\textwidth]{img/plot3.png}
\end{center}
\vspace{-2mm}
\caption{Same graph using dashed red line.}
\label{fig:plot3}
\end{figure}
\noindent
The graph can be plotted using green color and small dots rather than 
a solid or dashed line:

\begin{verbatim}
lb = "Dashed red line"
clf()
plot(x, y, 'g.', label = lb)
legend()
lab.show()
\end{verbatim}
The output is shown in Fig. \ref{fig:plot4}.

\newpage

\begin{figure}[!ht]
\begin{center}
\includegraphics[width=0.6\textwidth]{img/plot4.png}
\end{center}
\vspace{-6mm}
\caption{Same graph using dotted green line.}
\label{fig:plot4}
%\vspace{-5mm}
\end{figure}
\noindent
\noindent
Last let us stay with green color but make the dots larger:

\begin{verbatim}
lb = "Dashed red line"
clf()
plot(x, y, 'go', label = lb)
legend()
lab.show()
\end{verbatim}
\noindent
The output is shown in Fig. \ref{fig:plot5}.


\begin{figure}[!ht]
\begin{center}
\includegraphics[width=0.6\textwidth]{img/plot5.png}
\end{center}
\vspace{-6mm}
\caption{Same graph using large green dots.}
\label{fig:plot5}
%\vspace{-4mm}
\end{figure}
\noindent
For a complete list of options available for the {\tt plot} command, 
visit the Pylab page {\tt http:// www.scipy.org/PyLab}.

\section{Plotting General Planar Curves}

The concept of plotting based on two arrays of $x$ and $y$ coordinates
allows us to do much more than only plot graphs of functions of one variable.
We can easily plot more general curves such as circles, spirals and others.
Let us illustrate this on a spiral that this parameterized 
by 
$$
x(t) = t \cos(t), \ \ \ \ 
y(t) = t \sin(t)
$$ 
in the interval $(0, 10)$ for $t$. The complete code is

\begin{verbatim}
from pylab import *
from numpy import *
t = arange(0, 10, 0.05)
x = t*cos(t)
y = t*sin(t)
clf()
plot(x, y)
lab.show()
\end{verbatim}
The output is shown in Fig. \ref{fig:plot6}.

\begin{figure}[!ht]
\begin{center}
\includegraphics[width=0.6\textwidth]{img/plot6.png}
\end{center}
\vspace{-6mm}
\caption{Plotting a spiral.}
\label{fig:plot6}
\vspace{-4mm}
\end{figure}
\noindent


\section{Plotting Functions of Two Variables with WebGL}

For this functionality, your browser has to support WebGL (most of modern browsers do, 
with the exception of Internet Explorer). See the introductory tutorial {\em Welcome to NCLab!}
for detailed instructions on how to enable WebGL. NCLab offers the following simple way to plot 
functions of two variables:

\begin{verbatim}
from numpy import sin, sqrt, arctan

# Define intervals on the x and y axes.
x0 = 0.0
x1 = 10.0
y0 = 0.0
y1 = 10.0

# Define the corresponding divisions.
nx = 100
ny = 100

# Define a function:
def f(x, y):
    return sin(sqrt(x**2 + y**2))

# Render the surface using WebGL.
lab.surface((x0, x1, nx), (y0, y1, ny), f)
\end{verbatim}

\begin{figure}[!ht]
\begin{center}
\includegraphics[width=0.8\textwidth]{img/webgl.png}
\end{center}
\vspace{-2mm}
\caption{Plotting functions of two variable.}
\label{fig:webgl}
%\vspace{-1cm}
\end{figure}


\section{Plotting General Triangulated Surfaces with WebGL}

NCLab also supports plotting of general 3D surfaces via WebGL. The surface needs 
to be composed of linear triangles (same as for Open GL). There is a simple 
format in NCLab that one has to follow. Let us illustrate it on a sequence of very 
simple examples. First we will plot one triangle only. 

\begin{verbatim}
# Define color:
color = [0.4, 0.9, 0.6]

# Render a single triangle.
lab.visualize({'vertices': [0, 0, 1, 1, 0, 1, 1, 1, 1], 
               'indices': [0, 1, 2],
               'color': color})
\end{verbatim}
The numbers in {\tt color} are the RGB components scaled between 0 and 1. Further,
{\tt vertices} are triplets of coordinates of the grid points in 3D, {\tt indices}
are also triplets but of integers. Each such triplet contains indices of the vertices
of one triangle. Last is the parameter {\tt color}. 

\begin{figure}[!ht]
\begin{center}
\includegraphics[width=0.6\textwidth]{img/tria1.png}
\end{center}
\vspace{-2mm}
\caption{Plotting a single triangle.}
\label{fig:tria1}
%\vspace{-1cm}
\end{figure}
\noindent
When the {\tt color} parameter is omitted, the plot will be colored 
using the $x$-coordinate as a function value. For illustration, let us plot the 
same triangle as above:

\begin{verbatim}
# Render a single triangle (omit the color parameter).
lab.visualize({'vertices': [0, 0, 1, 1, 0, 1, 1, 1, 1], 
               'indices': [0, 1, 2]})
\end{verbatim}
Output:

\begin{figure}[!ht]
\begin{center}
\includegraphics[width=0.6\textwidth]{img/tria2.png}
\end{center}
\vspace{-2mm}
\caption{Plotting a single triangle (color omitted).}
\label{fig:tria2}
\end{figure}
\noindent
Now let us plot two triangles. just to understand the data format a bit better.
We have 4 grid points, therefore the {\tt vertices} array contains 12 numbers
and the {\tt indices} array contains 6 numbers (three per triangle).

\begin{verbatim}
lab.visualize({'vertices': [0, 0, 0, 1, 0, 0, 0, 1, 0, 0, 0, 1], \
               'indices': [0, 1, 2, 0, 3, 1], \
               'edges': [0, 1],
               'color': color})
\end{verbatim}
Output:
\newpage
\begin{figure}[!ht]
\begin{center}
\includegraphics[width=0.6\textwidth]{img/tria3.png}
\end{center}
\vspace{-2mm}
\caption{Plotting two triangles.}
\label{fig:tria3}
\end{figure}
\noindent
Note the new parameters {\tt edges} that contains pairs of integer indices
that define edges using the array of vertices.\\

\noindent
There can be up to tens of thousands of triangles in more complicated surfaces.
This will work, but keep in mind that rendering large volumes of data takes
more time.

\section{Interested in 3D Geometrical Modeling?}
At last, if you are interested in 3D geometrical modeling or CAD, we invite you to 
explore PLaSM (Programming Language for Solid Modeling) that is available in 
NCLab. PLaSM can be imported in any Python worksheet by typing 

\begin{verbatim}
from pyplasm import *
\end{verbatim}
Comprehensive tutorial in PDF along with three videos are available on NCLab's
Tutorial page.



\end{document}






\end{document}
