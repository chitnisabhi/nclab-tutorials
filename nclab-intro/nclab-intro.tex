\documentclass{article}


%%%%%%%%%%%%%%%%%%%%%%%%%%%%%%%%%%%%%%%%%%%%%%%%%%%%%%%%%%%%%%%%%%%%%%%%%
\pagestyle{plain}                                                      %%
%%%%%%%%%% EXACT 1in MARGINS %%%%%%%                                   %%
\setlength{\textwidth}{6.5in}     %%                                   %%
\setlength{\oddsidemargin}{0in}   %% (It is recommended that you       %%
\setlength{\evensidemargin}{0in}  %%  not change these parameters,     %%
\setlength{\textheight}{8.5in}    %%  at the risk of having your       %%
\setlength{\topmargin}{0in}       %%  proposal dismissed on the basis  %%
\setlength{\headheight}{0in}      %%  of incorrect formatting!!!)      %%
\setlength{\headsep}{0in}         %%                                   %%
\setlength{\footskip}{.5in}       %%                                   %%
%%%%%%%%%%%%%%%%%%%%%%%%%%%%%%%%%%%%                                   %%
\newcommand{\required}[1]{\section*{\hfil #1\hfil}}                    %%
\renewcommand{\refname}{\hfil References Cited\hfil}                   %%
\bibliographystyle{plain}                                              %%
%%%%%%%%%%%%%%%%%%%%%%%%%%%%%%%%%%%%%%%%%%%%%%%%%%%%%%%%%%%%%%%%%%%%%%%%%

\usepackage{graphicx}

\pagestyle{empty}

\begin{document}

\large

\vbox{}
\begin{figure}[!ht]
%\hspace{-4mm}
\includegraphics[width=8cm]{img/logo.png}
\vspace{12mm}
\end{figure}

\begin{figure}[!ht]
\begin{center}
%\hspace{-4mm}
\includegraphics[width=14cm]{img/desktop-0.png}
\vspace{16mm}
\end{center}
\end{figure}

\centerline{\Huge \bf Welcome to NCLab!}

\vfill

\centerline{\Large www.nclab.com}

\newpage

%%%%%%%%%%%%%%%%%%%%%%%%%%%%%%%%%%%%%%%%%%%%%%%%%%%%%%%%%%%%%%%%%%%%%%%%%



\section*{}
\small
\input ../common/aboutnclab.tex

%\subsection*{Acknowledgement}
%This publication was created with the help of numerous freely 
%available web resources and tutorials related to Python, Scipy,
%Numpy, Pylab, Matplotlib, Sympy and other projects.

\normalsize

\newpage
%{\ }
\setcounter{tocdepth}{2}
\tableofcontents
%\pagestyle{plain}

\newpage

\pagestyle{plain}
\setcounter{page}{1}


%%%%%%%%%%%%%%%%%%%%%%%%%%%%%%%%%%%%%%%%%%%%%%%%%%%%%%%%%%%%%%%%%%%%%%%%%
\newpage

\pagestyle{plain}

\section{Welcome!}

NCLab is a popular Online STEM Laboratory that serves thousands of students, instructors and 
researchers all over the world. It helps its users complete projects in many STEM fields ranging from
symbolical and numerical mathematics, computer simulations in physics and chemistry, 3D computer 
aided design (CAD), computer programming, and web design to engineering-level computer simulations 
in various areas including solid mechanics, fluid dynamics, electromagnetics, neutronics and others. 
NCLab can be used {\bf free of charge for personal non-commercial purposes} such as private 
hobby or self-education, as well as for individual non-funded academic research. Institutions 
need to {\bf purchase a license} for a symbolical fee. 

\begin{figure}[!ht]
\begin{center}
\includegraphics[width=\textwidth]{img/outside.png}
\end{center}
\vspace{-2mm}
\caption{NCLab's home page.}
\label{fig:outside}
%\vspace{-0.6cm}
\end{figure}

\noindent
Please visit NCLab's home page {\tt http://nclab.com}. The page features a slide show 
illustrating various activities offered by NCLab, and it provides links to related resources such 
as NCLab Blog, Tutorials, Displayed Projects, and others. Sign up for NCLab Newsletter 
in order to receive information about new functionality, webinars, and related topics! 

\subsection{WebGL}

WebGL is one of the most amazing features of NCLab, and at the same time one of the most 
problematic ones, 
so let us provide a short explanation of what goes on. {\bf You do not need to worry about WebGL 
unless you work with 3D geometries, CAD models, or need to plot graphs of functions of two 
variables.} 

For those who do -- WebGL is a web browser analogy of OpenGL 
(hardware-accelerated 3D graphics) that provides amazing 3D graphics. The technology is 
still experimental and it is turned off in some web browsers by default. Often an upgrade of 
graphics drivers combined with adjusting your browser's settings are needed to make it work. 
WebGL should work in modern web browsers with the exception of Internet Explorer 
where it is not supported.

Because of WebGL problems, we provide a "Test WebGL" button on NCLab's home page. Click on it.
If you see a window similar to the one in Fig. \ref{fig:webgl2} then you are fine. \\

\begin{figure}[!ht]
\begin{center}
\includegraphics[width=0.5\textwidth]{img/webgl2.png}
\end{center}
%\vspace{-2mm}
\caption{WebGL Tester.}
\label{fig:webgl2}
\end{figure}
\noindent
Now you can click into the WebGL tester window and move 
your mouse while keeping the left button pressed to see WebGL in action.
However, you might well see an error message similar to the one shown in Fig. \ref{fig:nowebgl}.\\

\begin{figure}[!ht]
\begin{center}
\includegraphics[width=0.7\textwidth]{img/nowebgl.png}
\end{center}
%\vspace{-2mm}
\caption{Message saying that WebGL is not enabled.}
\label{fig:nowebgl}
\end{figure}

\noindent
Below are some of the most usual reasons why WebGL is not working:
\begin{enumerate}
\item {\em You are using Internet Explorer (IE)}. IE does not have WebGL support. If you are using this web browser, please install another 
one such as Chrome, Firefox, Safari etc. This takes only minutes (at most) and it is highly rewarding in terms 
of improved Internet experience. In fact, poor performance of IE was causing such problems to NCLab users that 
we eventually decided to banish IE. 
\item {\em Your browser is outdated.} We recommend that you upgrade your web browser to the newest version, this 
can never hurt. 
\item {\em Your browser has WebGL disabled.} Try the following:
\begin{enumerate}
\item Firefox: Type {\tt about:config} in the address bar and hit enter. Ignore a message about harming your 
      warranty. In the search bar type {\tt webgl} and double-click on {\tt webgl.force-enable} and 
      {\tt webgl.prefer-native-gl} to set them to {\tt true}. Restart your browser (reloading page is not enough).
\item Chrome: Type {\tt about:flags} in the address bar and hit enter. Ignore a message saying that these 
      experiments may bite. Use CTRL + F to search for WebGL and enable it. Restart your browser. 
\end{enumerate}
\item {\em Everything fails.} In this case, try to upgrade your graphics drivers. This should help. If you still have 
problems, go to the page {\tt www.webgl.org} and try to find help there. You can also google for "graphic cards that 
support webgl" which will give you and idea whether your graphic card s supported or not. 
\item As a general observation, WebGL works almost always in Safari on more recent Mac systems  
\end{enumerate}

\subsection{Creating an Account}

A login dialog where
users enter their username, password and institution code (relevant for institution-sponsored users
only) is located in the top-right corner. In order to create a new account, click on the link "Create 
account" which is next to "Username". The following window appears:\\


\begin{figure}[!ht]
\begin{center}
\includegraphics[width=0.6\textwidth]{img/create-account.png}
\end{center}
%\vspace{-4mm}
\caption{Creating new account.}
\label{fig:creacc}
\end{figure}
\newpage
\noindent
In this form, enter your preferred username, email address (to be used only if you ask
us to generate a new password for you), and password twice. Next:

\begin{itemize}
\item If you belong to an institution 
      that uses NCLab for teaching or research, you should know your institution's code. Enter it in the form, 
      along with your first and last names. Your name will be visible to an NCLab admin
      who will verify that you belong to the institution.
\item If you are using NCLab for your personal hobby or self-education, for activities not 
      related to any institutional use, then leave the last three lines empty. 
\end{itemize}
After accepting the Terms of Use and clicking OK, you are ready to login!

\subsection{After Login}

The interior of NCLab is similar to a standard computer desktop, except that the 
icons are different (see front page screenshot). Many users maximize the web browser 
window via F11 to make the desktop experience even more realistic.
If any member of any of your groups is in NCLab at the time you log in, the Groups icon 
are lighted green. If you do not like the default desktop background, it can be 
changed easily in Settings.

Besides Groups, the other icons represent:

\begin{itemize}
\item {\em File Manager}, used to manage files and folders, and to clone work published by other users.
\item {\em Programming} module provides programming in 
      \begin{itemize}
      \item Karel the Robot (famous educational programming language). 
      \item Python (high-level dynamical programming language). 
      \item Javascript (most popular language for web development). 
      \item CUDA (massively parallel programming on graphic cards).
      \end{itemize}
\item {\em Physics} module provides various physics simulation tools in Kinematics,
      Solid Mechanics, Fluid Dynamics, Electromagnetics, and other areas. 
\item {\em Math} module offers Fractal Explorer (interactive graphical application 
      to explore Fractals and generate beautiful fractal art), ODE module to explore 
      ordinary differential equations, and PDE module to explore partial differential 
      equations. 
\item {\em Web Design} module is an integrated graphical application where you can 
      enter HTML, CSS and JS code and edit your web page in real time. 
\item {\em CAD} module features the PLaSM library (Programming Language for Solid Modeling).
      The library is used through scripting and a GUI is currently in progress.
\item {\em Calculator} is a calculator! 
\end{itemize}
Chemistry module and other modules are on the way. 

\subsection{Groups and Chat}

In NCLab, users can form groups, see who is online, and chat in real time. In order to 
create a new group, click on the Groups icon. A window similar to the one shown 
in Fig. \ref{fig:groups} appears. This concrete screenshot shows four groups of the 
user "solin" -- the top one was created by Jordan, the other three by solin. The 
green light next to the last group indicates that someone in that group
is currently in NCLab. Before we go look who that is, note that you can create a
new group of your own by clicking on "New Group". In order to add people to your groups,
you need to know their usernames. The button "Who's online" allows you to see instantly 
all users in all your groups who are currently logged in. 

\begin{figure}[!ht]
\begin{center}
\includegraphics[width=0.5\textwidth]{img/groups.png}
\end{center}
%\vspace{-2mm}
\caption{Groups window, showing that someone in the last group is currently in NCLab.}
\label{fig:groups}
\end{figure}

\newpage
\noindent
Clicking on the last group 
reveals that the online users are Emily and Rick.
\begin{figure}[!ht]
\begin{center}
\includegraphics[width=0.5\textwidth]{img/groups-2.png}
\end{center}
%\vspace{-2mm}
\caption{Detail of the last group.}
\label{fig:groups-2}
\end{figure}

\noindent
Clicking on any user who has the green light next to him/her (except yourself),
you initiate a chat to that user. Or someone from your groups may contact you. 
The Chat window is shown in Fig. \ref{fig:chat}.

\begin{figure}[!ht]
\begin{center}
\includegraphics[width=0.4\textwidth]{img/chat.png}
\end{center}
%\vspace{-2mm}
\caption{Chat window.}
\label{fig:chat}
\end{figure}

\subsection{File Manager}

Clicking on the File Manager icon launches the application. Initially, you will not 
have any files or folders there (Fig. \ref{fig:fileman}).

\begin{figure}[!ht]
\begin{center}
\includegraphics[width=0.9\textwidth]{img/fileman1.png}
\end{center}
\vspace{-2mm}
\caption{Initial launch of File Manager.}
\label{fig:fileman}
\vspace{-1cm}
\end{figure}
\newpage
\noindent
However, this can be changed quickly. Click on the menu Project $\rightarrow$ Clone. This will
launch a window with many displayed projects from various areas of math, 
physics, chemistry, 3D computer aided design, and engineering simulations (Fig. \ref{fig:fileman2}).


\begin{figure}[!ht]
\begin{center}
\includegraphics[width=0.9\textwidth]{img/fileman2.png}
\end{center}
\vspace{-2mm}
\caption{List of displayed projects.}
\label{fig:fileman2}
\end{figure}
\noindent
You can view the contents of any displayed project by double-clicking on it. You can 
clone any project into your account by just clicking on it once, and then clicking on 
the button "Clone". Let's say that we want to work with Python. After cloning 
the project "Python - Intro", we will see the following in the File Manager:

\newpage

\begin{figure}[!ht]
\begin{center}
\includegraphics[width=\textwidth]{img/fileman3.png}
\end{center}
%\vspace{-2mm}
\caption{After cloning the project "Python - Intro".}
\label{fig:fileman3}
\end{figure}
\noindent
Now by clicking on the project, it launches in a new window, as
shown in Fig. \ref{fig:fileman4}.

\newpage

\begin{figure}[!ht]
\begin{center}
\includegraphics[width=\textwidth]{img/fileman4.png}
\end{center}
%\vspace{-2mm}
\caption{Launching project "Python - Intro".}
\label{fig:fileman4}
\end{figure}
\noindent
The worksheet contains {\em input cells} where you can enter or edit computer
code, and descriptive {\em text cells}. The contents of text cells can be 
edited after clicking on the cell. The text is be formatted using Restructured
Text (RST) format. You can click on the blue arrow button to evaluate all input 
cells in the worksheet (hold on with that for a moment), or you can evaluate 
each input cell individually by clicking on the "run" right under it. 

Let us evaluate the first input cell that contains just {\tt 1 + 3}. This will send its contents 
to the server where it will be parsed using a Python interpreter, and the 
output will be sent back to your browser and displayed almost instantly in
a new yellow {\em output cell} (Fig. \ref{fig:fileman5}).

\newpage

\begin{figure}[!ht]
\begin{center}
\includegraphics[width=\textwidth]{img/fileman5.png}
\end{center}
%\vspace{-2mm}
\caption{Output is displayed in yellow {\em output cells}.}
\label{fig:fileman5}
\end{figure}
\noindent
We recommend that you spend some time experimenting with the 
File Manager's menu and this worksheet. Try to:
\begin{itemize}
\item Insert new input cells below and about an existing input cell.
\item Insert new text cells below and about an existing input cell.
\item Remove cells by clicking on "remove" under tehm.
\item Create a new folder "Python" and drag your cloned Python 
      project in there. 
\item Rename the file and folder.
\item Duplicate your project via "Save as" under a different name.
\item Synchronize your cloned project with the original via the 
      "Synchronize" button. Beware though - synchronizing discards 
      all your changes that you made to the project since cloning. 
\item Access a Python Tutorial via the Help button.
\end{itemize}






%\subsection{WebDesign}
%
%The Web Design kit is an integrated environment for the design of web pages.
%The graphical application contains three input panels for HTML, JS and CSS, and 
%one output panel. The web page is changing in real time when any of the three 
%input panels are edited. The output panel can be detached and magnified for better 
%viewing comfort. Users belonging to an institutions and 
%users operating NCLab in Full Version can have their web pages hosted.
%
%\begin{figure}[!ht]
%\begin{center}
%\includegraphics[width=\textwidth]{img/progr6.png}
%\end{center}
%%\vspace{-2mm}
%\caption{Web Design kit - illustrative screenshot.}
%\label{fig:progr6}
%\end{figure}
%\noindent
%More information is available in the Help section.

\section{Physics Module}

The Physics module contains a quickly growing number of graphical applications 
from many different areas including kinematics, solid mechanics, 
fluid dynamics, electromagnetics, and others.

\subsection{Kinematics}

Kinematics deals with moving objects.  
Classical application from this field -- Projectile Motion -- enables computing 
the trajectory of a flying projectile either with or without considering 
air friction. Illustrative screenshot is shown in Fig. 
\ref{fig:kinem1}. 

\begin{figure}[!ht]
\begin{center}
\includegraphics[width=\textwidth]{img/kinem1.png}
\end{center}
%\vspace{-2mm}
\caption{Projectile motion.}
\label{fig:kinem1}
\end{figure}
\newpage
\noindent

\subsection{Solid Mechanics}

Solid mechanics deals with deformation of solid objects. NCLab 
offers an Elasticity module that allows the user to define an
arbitrary 2D object, define its material properties (density, Young modulus, 
Poisson ratio), prescribe displacements and acting forces on the boundary, 
and calculate displacements and stresses inside the object. This application  
uses the Hermes library (http://hpfem.org/hermes) for the finite element 
analysis (FEA) part. Illustrative screenshot is provided in Fig. \ref{fig:elast1}.
\begin{figure}[!ht]
\begin{center}
\includegraphics[width=\textwidth]{img/elast1.png}
\end{center}
%\vspace{-2mm}
\caption{Elasticity analysis of a hollow steel pipe of square cross-section that is loaded from above
         (shown is Von Mises stress).}
\label{fig:elast1}
\end{figure}
\noindent
The simulation process consists of several steps: geometry definition, 
mesh generation, problem definition, computaion, and postprocessing. 
The application will walk you through these steps, and Help 
is available for each of them.

\subsection{Electromagnetics}

The Electromagnetics module contains at the moment two modules: an axisymmetric 
capacitor model and a general electrostatics model. In both cases one calculates 
the distribution of the electric field surrounding charged objects. Boundary 
conditions can be prescribed voltage on some part of the boundary and / or 
prescribed electric charge density on another part of boundary. Two illustrative 
screenshots of the capacitor model are provided below.
\begin{figure}[!ht]
\begin{center}
\includegraphics[width=\textwidth]{img/capac1.png}
\end{center}
%\vspace{-2mm}
\caption{Plate capacitor - axisymmetric model.}
\label{fig:capac1}
\end{figure}
\noindent

\newpage

\begin{figure}[!ht]
\begin{center}
\includegraphics[width=\textwidth]{img/capac2.png}
\end{center}
%\vspace{-2mm}
\caption{Capacitor model - results.}
\label{fig:capac2}
\end{figure}
\noindent


\subsection{Fluid Dynamics}

The Fluid Dynamics module contains a model of draining a tank filled with water. 
Illustrative screenshot is shown in Fig. \ref{fig:tank1}.

\newpage

\begin{figure}[!ht]
\begin{center}
\includegraphics[width=\textwidth]{img/tank1.png}
\end{center}
%\vspace{-2mm}
\caption{Draining a tank with water.}
\label{fig:tank1}
\end{figure}



\section{Math Module}

The Math module contains several graphical applications including the Fractal 
Explorer, and applications for ordinary differential equations (ODE) and partial 
differential equations (PDE). Besides this, symbolic and numerical math 
as well as graph theory are available through Python libraries Sympy,
Numpy, Scipy, Pylab, NetworkX and others. These will be discussed in the following 
documents.

\subsection{Fractal Explorer}

The Fractal Explorer allows you to get acquainted with beautiful 
fractal structures of the Mandelbrot and Julia sets. Tutorial 
is available via the "Tutorials and Videos" link on NCLab's front page. 
Sample screenshot showing a Julia set is below:

\begin{figure}[!ht]
\begin{center}
\includegraphics[width=0.7\textwidth]{img/julia.png}
\end{center}
%\vspace{-2mm}
\caption{Julia fractal.}
\label{fig:julia}
\end{figure}




\subsection{ODE}

Part of the Ordinary Differential Equations (ODE) section are advanced Java applets JODE by Marek 
Rychlik (University of Tucson) for plotting slope fields and solutions 
of two-ODE and three-ODE systems. Illustrative screenshot of the slope 
field application is below.

\newpage

\begin{figure}[!ht]
\begin{center}
\includegraphics[width=\textwidth]{img/jode1.png}
\end{center}
%\vspace{-2mm}
\caption{Slope fields with JODE.}
\label{fig:jode1}
\end{figure}


\subsection{PDE}

The Partial Differential Equations (PDE) section contains a Finite Element Analysis (FEA)
application for solving general linear second-order equations with constant coefficients.
Illustrative screenshot is shown below.
\newpage
\begin{figure}[!ht]
\begin{center}
\includegraphics[width=\textwidth]{img/pde1.png}
\end{center}
%\vspace{-2mm}
\caption{Solving linear second-order PDE.}
\label{fig:pde1}
\end{figure}

\section{Solid Modeling (CAD)}

Solid Modeling is the basic principle of CAD systems -- complicated geometries 
and shapes are created using simple objects that can be translated, rotated and scaled. 
One can also create their unions and intersections, and subtract objects 
from each other. NCLab provides Solid Modeling via PLaSM (Programming Language for Solid 
Modeling). 

Professional CAD systems such as SolidWorks or AutoCAD are aimed at maximizing the engineer's
productivity. Therefore, their functionality is highly automated. In contrast to them, PLaSM 
is a scripting language that allows the user to gain a better insight into the underlying 
geometrical operations, as well as create parameter-dependent designs. 
For illustration, the following short script renders a metal cube:
\begin{verbatim}
from pyplasm import *
color = [0.9, 0.9, 0.9]
c = CUBE(1.0)
lab.view(c, color)
\end{verbatim}
Fig. \ref{fig:plasm1} shows more advanced models of a 3D gear and 
a temple:

\begin{figure}[!ht]
\begin{center}
\includegraphics[width=\textwidth]{img/plasm1.png}
\end{center}
%\vspace{-2mm}
\caption{Sample 3D models created with PLaSM.}
\label{fig:plasm1}
\end{figure}
\noindent
To learn more, read the PLaSM tutorial and watch a three-part YouTube video!

\section{Parallel computing with GPUs}

GPU stands for {\em Graphics Processing Unit} and massively parallel computing with GPUs
is one of the hottest new trends in Computational Science. The main advantage 
of GPUs over standard CPUs (Central Processing Units) is that on the same price 
level, GPUs provide much larger number of cores and Flops (number of floating 
operations per second) compared to CPUs.
 
CUDA is an Nvidia language for GPUs. The CUDA worksheet provides programming via 
PyCUDA (Python wrappers to CUDA). Tutorial to PyCUDA is available via the link "Tutorials and Videos" 
on NCLab's front page and via the Help button, and there are around 30 
displayed projects where you can learn both basic and advanced aspects of CUDA
programming. 

\begin{figure}[!ht]
\begin{center}
\includegraphics[width=\textwidth]{img/progr4.png}
\end{center}
%\vspace{-2mm}
\caption{CUDA programming - illustrative screenshot.}
\label{fig:progr4}
\end{figure}
\noindent

\section{Appendix - What is the {\em Cloud}?}

Using cloud computing or cloud computers sounds like a rocket science, but it is not,
and it really changes things for the better. Chances are that you are using 
it already, without even knowing. For example, if you are using free email services of Gmail, 
Hotmail, Yahoo or another such provider, then your data are stored on the cloud. If you are 
using free Google docs to acces your Word or Excel files from anywhere, you are using the cloud
as well. 

\subsection{Basic Facts}

In the IT sense of the word, and with a bit of simplification, the {\em cloud} 
is a pool of interconnected computers. They come in different sizes depending on the provider - 
large cloud facilities have millions of computers but some companies are running 
their own private clouds with relatively few ones. The computers in large clouds
are not very similar to your desktop PC -- they are stripped off many unnecessary 
things, compacted, and interconnected into a powerful grid. Their processors 
typically contain multiple {\em computing cores} that can share the same memory.
Such {\em multicore processors} are much more efficient compared to (clusters 
of) PCs with equivalent number of cores.

The main advantage of the cloud is its {\em elasticity}. Upon your request, the
provider will assemble for you an {\em instance} with the parameters that 
you need. An instance means a computer with given parameters (number of processors, 
hard disk size, memory). An instance is created within a minute or so, and the user can upgrade 
or downgrade it dynamically, depending on the actual needs. After the instance is 
no longer needed, it is dissolved and the same hardware is used to build instances 
for other users. The following is a realistic example of how such instances may 
look like (just the name of the provider was omitted):\\

\begin{center}
\begin{tabular}{|l|l|l|l|}
\hline
{\bf Small instance} & 1.7 GB of memory & 1 core & 160 GB hard disk \\
\hline
{\bf Medium instance} & 7.5 GB of memory & 4 cores & 850 GB hard disk \\
\hline
{\bf Large instance} & 15 GB of memory & 8 cores & 1690 GB hard disk \\
\hline
\end{tabular}
\end{center}

\vspace{4mm}
\noindent
Importantly - these three instances are created by grouping the same resources together 
in different ways. 
It is quite interesting to look at the prices:

\begin{center}
\begin{tabular}{|l|l|}
\hline
{\bf Small instance} &	\$0.085 per hour\\
\hline
{\bf Medium instance}&	\$0.34 per hour	\\
\hline
{\bf Large instance}&	\$0.68 per hour\\
\hline
\end{tabular}
\end{center}

\vspace{4mm}
\noindent
With such low prices, literally anyone has now access to 
a powerful computer -- that otherwise would cost many thousands of dollars -- 
for just cents per hour. This brings new wonderful opportunities to ordinary users
and educators.

\subsection{Software as a Service (SaaS)}

Another big change that cloud computing brings is in how software is handled. 
Traditionally, one would buy a software in a big box with a small CD in it, 
and install it on one's computer. This model is now being challenged by a new 
approach called {\em Software as a Service (SaaS)}. The user does not have 
to own a copy of the software physically. Instead, the software is running 
on a remote server and accessed by users over the Internet. Typically, paying 
for the access is much less expensive compared to buying the software. If you like,
think about it like watching a movie on Netflix vs. buying a DVD. 

\subsection{Access from Mobile Platforms}

Last but not least, since the software is running on the cloud, the user's hardware 
does not matter so much anymore. The only really important things is to be able to 
run a web browser and access the Internet. This can be done on anything between smart 
phones, tablets, netbooks, laptops, and desktop computers. 

\subsection{Some Myths}

Very often, cloud computing is mentioned in the context of large 
computations in science, engineering, finance, and other fields. This is because there are many 
cloud providers and they need to show off. But in reality, the clouds are mostly
busy processing small tasks of ordinary users. In that case, you may say, a standard 
office PC should be enough. Not exactly. Once you are not in your office, for example
while traveling, it is difficult to reach your office PC and do things as usual. 
The cloud changes that -- it allows you to access your data 
from everywhere, any time. Once you get used to it, there is no way back. 

\subsection{NCLab and K-12 Education}

NCLab is an Online STEM Laboratory that provides 
instant access to computer simulations in physics and chemistry, 3D CAD design, 
engineering-level computer modeling, and scientific computing. You do not have 
to install any software -- NCLab is automatically available in any classroom that 
has Internet access. In contrast to traditional educational software products, 
users can access their accounts and work from anywhere and at any time.
Students can start a problem at school, finish the rest at home, and notify 
the instructor about the completion with one mouse click. Most importantly, 
however, NCLab invites K-12 teachers and students to discover new exciting 
areas that were out of their reach before.


\section{Review Questions}

\begin{enumerate}
\item What is {\em Cloud}?
\begin{enumerate}
\item[A1] Large pool of interconnected computers.
\item[A2] Local LAN network in an organization or school.
\item[A3] Computer that is equipped with keyboard but not monitor.
\item[A4] Company that manufactures large computer clusters.
\end{enumerate}
\item Name at least one free cloud service.
\begin{enumerate}
\item[A1] Gmail.
\item[A2] United States Postal Service
\item[A3] Fedex
\item[A4] UPS
\end{enumerate}
\item What does the abbreviation {\em SaaS} stand for?
\begin{enumerate}
\item[A1] Simple and always Stable.
\item[A2] Synchronized and automated Service.
\item[A3] Software as a Service.
\item[A4] Super addictive and Sweet.
\end{enumerate}
\item What is WebGL?
\begin{enumerate}
\item[A1] Webinar on Grenade Launchers.
\item[A2] Advanced 3D graphics in the web browser.
\item[A3] Web page of the Girls Life magazine.
\item[A4] Abbreviation for "Good Luck" among web designers.
\end{enumerate}
\item What is the main advantage of GPUs over CPUs?
\begin{enumerate}
\item[A1] GPUs are smaller and lighter than CPUs.
\item[A2] GPUs are cheaper than CPUs.
\item[A3] GPUs provide much more computing power than similarly priced CPUs.
\item[A4] GPUs can operate without electricity.
\end{enumerate}
\item Name one of the most famous fractal sets.
\begin{enumerate}
\item[A1] Mandelbrot.
\item[A2] Zuckerberg.
\item[A3] Heisenberg.
\item[A4] Heidenhain.
\end{enumerate}
\item What is Solid Modeling?
\begin{enumerate}
\item[A1] Method to create arbitrary geometries via simple objects.
\item[A2] Computer modeling in Solid Mechanics.
\item[A3] Computer modeling in Solid State Physics.
\item[A4] Modeling with {\em Solid} (a computer modeling software).
\end{enumerate}
\end{enumerate}

\end{document}
